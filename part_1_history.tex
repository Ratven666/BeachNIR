

% \subsection{Исторический контекст формирования системы взаимодействия человека с прибрежными территориями}

\subsection{Исторический контекст формирования системы \mbox{взаимодействия человека с прибрежными территориями}}


\subsubsection{Глубокая древность}

Расселение человека вдоль водных объектов с начала существования его как биологического вида было жизненной необходимостью.
Отсутствие средств аккумуляции, переноса и хранения воды грозило человеку, как впрочем и любому животному, очевидными угрозами для здоровья и жизни в случае его отрыва от источника пресной воды.
Необходимость присутствия водных объектов в пешей доступности превратила береговые линии в естественный ориентир, вдоль которого первые люди провели свое расселение.
Экстенсивный характер природопользования первых людей накладывал на них необходимость обеспечения каждого из субъектов большой территорией, что в свою очередь послужило стимулом к расселению человечества по всем пригодным для естественного проживания территориям Земли.  

Тысячелетия эвристического поиска позволили человечеству перейти от присваивающей к производящей модели экономики совершив при этом первую в истории революцию -- неолитическую.
Формирование сельского хозяйства позволило перейти к оседлому образу жизни, сформировав не только базис для роста популяции, но и прибавочный продукт, позволивший в перспективе обеспечить разделение труда.
Развитие этих процессов определило удобство группового проживания людей на локальных участках местности сформировав первые города, определившие своим названием само явление \enquote{цивилизации}.

Однако примитивность древних технологий позволяла лишь использовать ограниченные природные блага, локализуясь в местах их естественного происхождения.
Первые цивилизации: Месопотамии в междуречье Тигра и Ефрата, египетские царства Нила, древние китайские цивилизации в долине рек Янцзы и Хуанхэ, Хараппская цивилизация в долине Инда, цивилизации Амазонии и Мезоамерики -- все они стали возможны благодаря естественным иловым удобрениям земель вследствие периодичных разливов прилегающих к ним рек.
Блага такого соседства были для людей очевидны, однако в противовес этому древние люди подвергали себя риску экстремальных разливов рек, дошедших до нас в общераспространенном и архетипическом мифе о \enquote{великом потопе}.
Этот генетический страх беззащитности перед водной стихией до сих пор живет в глубине нас, определяя эмоциональную вовлеченность, начиная от современных религиозных текстов и заканчивая детскими сказками о Мумми-троллях.
Все эти сюжеты, определяя надежду на возрождение после мировой катастрофы, стимулировали людей искать собственные способы защиты от природных сил.

Осознав свою силу над природой, человечество начало трансформировать окружающий себя водный каркас, производя искусственную мелиорацию прилегающих земель и повышая тем самым урожайность своих земель.
Произведенные избыточные продукты стали условием обмена и бартерной торговли между соседствующими культурами и народами.
С этого момента и до наших дней водный транспорт является наиболее экономически эффективным в плане доставки грузов.
Начав от сплавов по рекам, каботажные плавания вдоль берегов сшили разрозненные  города и полисы Средиземноморья между собой, став одним из условий существования Великой Римской Империи.

Не являясь в этом первыми, именно римляне достигли недосягаемых для своего времени высот в вопросах управления водными ресурсами.
Лишь вскользь упомянув выходящий за рамки данной работы факт сооружения акведуков, сконцентрируемся на примерах морских гидротехнических сооружений.
Известный опыт сооружения морской дамбы в период Третьей Пунической войны позволил в итоге римлянам замкнуть кольцо осады Карфагена, отчасти повторив опыт Александра Македонского при осаде Тира.
Освоив технологии бетонных работ, римский гений применил его при сооружении пирсов в порту Косса, волнозащитных моллов в Остии и Цезарии, однако, несмотря на монументальность выполненных работ, природа побеждала и разрушала инженерные объекты \cite{mccann1987roman}.

Климатические изменения, наравне с кризисом рабовладельческой экономической модели экономики, погрузил человечество с темные средние века.
Несмотря на то, что необходимость гидротехнических работ не была забыта при феодальной формации качественный прогресс в этом вопросе был достигнут лишь при наступлении Нового времени. 

\subsubsection{Старая Голландия и Новое время} 

Одним из наиболее захватывающих и вдохновляющих примеров систематической работы над трансформацией морских прибрежных территорий является Нидерландский опыт, длившийся на протяжении двух тысячелетий и не останавливающийся в наши дни.
Подробный очерк, описывающий основные вехи этого противостояния человека и моря, описан в прекрасной работе \cite{Alahverdi2016}, ключевые фрагменты которой будут определять дальнейший текст.

Антропогенное влияние человека, выражавшееся в вырубке лесов под личные нужды и высвобождении земель под пахотные угодья, стало катализатором эрозионных разрушительных процессов, усугублявшихся нагонными волнами из Северного моря и общим повышением уровня Мирового океана.
Море стало неприклонно забирать землю.
Начиная с VI века до нашей эры прибрежные жители стали формировать искувсственные возвышения для защиты своих жилищ, развивая свой успех последующим сооружением дамб и насыпей.
Опуская подробности этой постоянной, каждодневной борьбы, состоящей из череды сменяющейся побед и поражений человека перед природой, можно выделить переломную точку в катастрофическом наводнении 1421 г.
Считается, что именно оно послужило стимулом к переходу от разрозненных персональных усилий к централизации мероприятий по терраформированию прибрежных территорий \cite{Alahverdi2016}.

Лишь к XVI веку, благодаря разработке инновационных ветряных мельниц и адаптации их для откачки воды, удалось приступить к последовательному осушению польдеров.
Человек перешел от защиты к нападению и в результате победил (рисунок \ref{pic:Windmill}).

\begin{figure}
    \begin{center}
	\includegraphics[width=150mm]{fig/The_Windmill_at_Wijk_1670_Ruisdael.jpg}
	\caption{-- «Мельница около Дюрстеде» картина Якоба ван Рейсдала, 1670~г.}
	\label{pic:Windmill}
    \end{center}
\end{figure}

На протяжении следующих веков сооружаемые дамбы служили как инструментом активной обороны в многочисленных европейских войнах, так и источником опасностей для жителей Северной Голландии, не исчерпавшим себя до конца вплоть до наших дней.
Известные своими ужасающими последствиями наводнения Зейдерзе (1916 г.) и прорывы дамб в дельте Рейна и Мааса (1953 г.) стали катализатором развития всемирно известной голландской геомеханической и гидротехнической школы, опыт который позволил предсказать и избежать многочисленные техногенные и природные катастровы по всему миру.

Современный Нидерландский опыт будет не раз упоминаться далее при анализе научных источников по рассматриваемой теме.
Искупить недостойно скудное и поверхностное перечисление Нидерландского опыта гидротехнических работ, в текущем разделе, можно дополнительным акцентированием внимания на то, что именно им принадлежит заслуга перехода от сугубо эвристического опыта взаимодействия с прибрежными территориями к сугубо научному, и пока еще естественному для нас, восприятию окружающего мира.
Сформированная в период Нового времени философия строго научного познания сформировала весь окружающий нас мир.
Символично, что именно из Голландии силой, энергией и волей Петра I свет этого знания перешел на территорию России, об опыте которой речь пойдет далее.

\subsubsection{Гидротехнический опыт Российской империи}  

Важнейшей вехой в истории освоения водных объектов стало правление Петра I, неразрывно связавшее будущее России с флотом, морем и западным миром.
К моменту его воцарения Россия не имела выхода к внешним морям, кроме выхода к Белому морю у г. Архангельска.
Удаленный замерзающий на большую часть года Архангельский порт уже к тому моменту не мог полностью удовлетворить потребности России во внешней торговле \cite{GapeevKononov2009}, что потребовало проведения срочных реформ.

Опыт и специалисты привезенные Петром I из Голландии на русскую землю позволили в считанные годы сократить экономическое отставание от соседних государств, создать первый русский флот и победить Швецию в Северной войне.
Период правления Петра I -- это период форсированного развития, период проб и ошибок -- начавшийся с ограниченного успеха и утраты Азова и Таганрогской гавани и увенчавшийся успехом градостроительного опыта в дельте Невы.
Кронштадтские укрепления на острове Котлин, галерная гавань на Васильевском острове и торговый порт на его стрелке стали примером первых берегоукрепительных и дноуглубительных работ, создания новых насыпных территорий на территории известного нам Санкт-Петербурга \cite{ChuneevaIzbasarova2021}.
Большие инфраструктурные проекты примерами которых являются Вышневолоцкая, Мариинская и Ивановская канальные системы были начаты при нем и закончены его преемниками \cite{GapeevKononov2009}.

Недолгое, но яркое правление Петра I прошло под девизом скорости и рациональности в ущерб классической монументальности ассоциирующейся с периодом начавшимся во второй половине XVIII века.
Эта \enquote{эпоха гранита}, оставленного отступившим оледенением дошла до нас в виде набережных сковавших берега Невы и городских каналов, защитивших постройки от наводнений и оползневых процессов, повысив при этом допустимую плотность застройки города.

Закрепившись на Балтийском море военные, а вслед за ними и гражданские силы, были направлены на юг к прибрежным территориям Черного и Азовского морей.
Плоды побед в Русско-Турецких войнах расцвели в формах русских портов Одессы и Новороссийска.

Эпоха пара, бетона и механизации производств, стартовавшая в России со второй половины XIX века, превратила прибрежные территории в индустриальные, связанные между собой не только водным и канальным транспортом, но и железными дорогами.
Созданные производства не могли существовать без воды и людей, определив тем самым как свое непосредственное расположение, так и урбанизированный ландшафт доходных домов, окружающих исторический центр городов \cite{BelyaevaBelyaev2021}.

В целях совершенствования качества прибрежных территорий также производились масштабные дноуглубительные работы (рисунок \ref{pic:МорскойКанал1885}), потребовавшие для своего проведения, помимо использования производительной техники, подготовки узкоспециализированных специалистов гидротехников и геодезистов, повысив тем самым доступный жителям уровень технического образования.

\begin{figure}
    \begin{center}
	\includegraphics[width=150mm]{fig/МорскойКанал1885}
	\caption{-- План Морского канала и порта на Гутуевском острове, 1885~г.~\cite{ChuneevaIzbasarova2021}}
	\label{pic:МорскойКанал1885} % название для ссылок внутри кода
    \end{center}
\end{figure}


Развивающиеся железные дороги укрепили связь между западной Россией и Дальним Востоком. 
В 1897 г. началась закладка первых бетонных массивов для устройства причалов портовых сооружений в бухте Золотой Рог Владивостока, велись дноуглубительные работы вдоль фарватера Амура.
В 1898 г. Россия получает в аренду Квантунский полуостров и начинает работы по созданию Порт-Артурской крепости и коммерческого порта Дальний, утраченных в скором времени в результате поражения в Русско-Японской войне.

Неким «Арктическим хвостом» этого периода истории стало обустройство в разгар Первой мировой войны незамерзающего порта в Мурманске, как реакции на риски морской блокады со стороны Балтийского моря.

Подводя итог этому историческому периоду, нельзя упрекнуть государственных и промышленных деятелей в пренебрежительном и легкомысленном отношении к прибрежным территориям.
Несмотря на гигантскую площадь Российской империи доступные к использованию морские прибрежные территории оставались весьма ограниченными: замыкаясь внутренними морями Новороссии, Балтийским морем, полуокруженным скандинавскими странами, суровыми северными портами Архангельска и Мурманска, молодыми городами Дальнего Востока.
Вызовы освоения северного морского пути перешли по наследству Советской власти, благодарно принявшей их наравне с результатами обширного накопленного гидротехнического опыта, ценимого как внутри страны, так и и за ее пределами \cite{Peterson2016}.

\subsubsection{Советский период}

Советский период отечественной истории неизбежно ассоциируется с прогрессивными инновациями во всех областях жизни общества.
Не явилась исключением и область освоения прибрежных территорий морей и рек нашей страны.

Общественный характер производства и использования ресурсов позволил Советскому государству перейти от стихийного и сугубо утилитарного использования прибрежных территорий к планомерному, последовательному освоению.
Фактически это позволило советской архитектурной школе рассматривать прибрежные гидротехнические объекты как часть архитектурного ансамбля и фасада прибрежных городов.
В работе \cite{Rubleva2022} подробно описана история развития концепции «Морского фасада» Ленинграда, предполагающую намыв территорий на Васильевском острове, спрямление берегов и строительство гранитных набережных, формирующих новое «лицо» города, обращенное к Балтике.
Элементы этого опыта были использованы советскими архитекторвами участвовавшими в разработке проектов восстановления разрушенных в период Великой Отечественной Войны черноморских городов \cite{VasilievOvsyannikova2019}.

Непрекращающийся рост советской экономики определил необходимость расширения портовых мощностей, реализовавших себя в создании новых глубоководных терминалов, примерами которых могут служить Находка и Ильичевск \cite{Sabaydash2023}, разгружающие порты Владивостока и Одессы соответственно.
Воля и смелость советского человека основала города и поселки Арктики: Диксон, Тикси, Певек, обеспечивая возможность полярной навигации и контроль наших северных границ.

Уникальность перечисленных мероприятий не позволила бы им состояться без привлечения мощной гидротехнической школы сконцентрированной внутри созданных отраслевых институтов: Союзморниипроекта, Ленморниипроекта, Новоморниипроекта и пр.

С 1960-х годов фокус внимания советской гидротехники распространяется на комплексное использование морских берегов, обеспечивая их рекреационную функцию.
Советские инженеры (В.П. Зенкович и др.) разработали теорию «свободных пляжей» как наилучшей защиты берега, что позволило сохранить курортную функцию городов при интенсивной эрозии береговой линии.
Внедрение унифицированных железобетонных элементов для берегоукрепления позволило обеспечить масштабное строительство бун, волноломов и искусственных пляжей в Сочи, Крыму и Прибалтике \cite{Kuklev2003}.

Однако высокие темпы развития прибрежных морских территорий не обошли и досадные ошибки, допущенные при стратегическом планировании этих территорий.
Разобщенность между ведомствами которым подчинялись порты (Минморфлоту) и горисполкому, определявшему городское устройство, приводило к дисбалансу в принимаемых решениях и конфликтах \enquote{город-порт}.
Пример Мурманска наглядно демонстрирует, к чему приводит победа в подобных спорах порта, фактически отрезавшего доступ для жителей к береговой линии.

Подводя итог очередному рассматриваемому периоду отечественной истории, можно с одновременной гордостью и грустью признать его \enquote{золотым веком} в области науки, культуры и практики в отношениях между человеком и природой морских прибрежных территорий.
К сожалению, богатое наследство гидротехнических сооружений, оставленных нам предками, к настоящему времени ведет одинокую и обреченную борьбу со стихией без должной поддержки со стороны властей.
Рассмотрению подробностей развития этой борьбы посвящен заключительный этап исторического обзора.

\subsubsection{Наше время}

Распад Советского Союза в 1991 году не только радикально изменил внутреннее устройство входящих в него республик, но и внешние, в том числе и морские, границы ставших независимыми стран.
Потеря значительной части портовой инфраструктуры на Балтике и Черном море, преобразование внутреннего Азовского и условно внутреннего Каспийского моря в международные акватории трансформировали условия для освоения прибрежных территорий. 
За три с лишним десятилетия постсоветского периода Россия прошла путь от острого кризиса в управлении прибрежным комплексом до восстановления и формирования новой стратегии освоения морских пространств.

Одной из краеугольных причин, формирующих проблемы в сфере использования прибрежных территорий, стало усложнение модели взаимодействия между выгодопользователями в рамках рыночных отношений, усугубившее проблему взаимодействий \enquote{город-порт}, добавив в нее еще и \enquote{собственника}.
Превратив прибрежные территории в условиях рыночной экономики в ценный земельный ресурс для жилого строительства \cite{ShcheglovaDokhov2025}.
Печально, что одним из главных пострадавших в этих спорах стал безмолвный городской обыватель, фактически вывеленный из участия в принятии решений по использованию прибрежных территорий.

Вдоль береговых линий было запущено \enquote{право сильного} и богатого.
В результате возник парадокс -- с формальной точки зрения Водный кодекс РФ \cite{WaterCodeRF} гарантирует свободный доступ к воде, запрещая приватизацию участков и строительство в границах 20 метров от берега, очевидно допуская все это за этим буфером.
С другой стороны, согласно КоАП РФ ст. 8.12.1 штрафы за нарушение режима водоохранной зоны и закрытие доступа граждан к береговой полосе предусматривает штраф 3000–5000 руб. для гражданских лиц.
Сложно представить, но до сих пор, несмотря на суровость наказания, вдоль берегов нашей родины существуют отдельные лица, идущие на риск подобных взысканий, де факто присваивая себе общую землю.

Расщепление внимания власти и попустительство при привлечении землепользователей к ответственности привело к абразии, оползням и размывам берегов, что привело к сокращению ширины пляжей и отступанию берегового уступа \cite{Kuklev2003}.
Наглядный пример, отражающий взаимодействие отдыхающих с остатками пляжей Черного моря, представлен на рисунке \ref{pic:Пляж_в_Лазоревском}.

\begin{figure}
    \begin{center}
	\includegraphics[width=150mm]{Пляж_в_Лазоревском.png}
	\caption{-- Берег на северной окраине пос. Лазаревское (г.~Сочи)~\cite{SochiCoastal2018}}
	\label{pic:Пляж_в_Лазоревском}
    \end{center}
\end{figure}

Прогрессивные действия по наведению порядка и культуры использования прибрежных территорий начались только в XXI веке.
Освоение нефтегазовых шельфов в рамках проектов Сахалин-1 и Сахалин-2, развитие Сан\-кт-Петер\-бург\-ского большого порта, инфраструктурные проекты Владивостока на острове Русский были оттенены масштабом подготовки к Сочинской зимней олимпиаде 2014 года.  
Олимпийский проект сделал Сочи самым дорогим в истории и радикально усилил туристско-рекреационный потенциал региона \cite{Mishulina2014}.

После воссоединения Крыма с Россией в 2014 году главным инфраструктурным проектом стал \enquote{Крымский мост}, строительство которого было завершено в мае 2018 года.
В то же время был запущен, длящийся до сих пор, процесс перераспределения туристического потока с заграничных курортов \cite{Ivanenko2024}.
По состоянию на 2020 год Причерноморье имеет самую высокую в России плотность туристической инфраструктуры.
В ходе оценки туристического потенциала Краснодарского края была выполнена масштабная классификация пляжей: более 70 пляжных зон официально проклассифицированы в Причерноморье, только в Сочи функционирует свыше 130 пляжей.
Азовское побережье (применительно к границам РФ 2021 года) протяженностью 550 км ежегодно принимало более 2,5 млн туристов \cite{Volkova2021}. 
При этом порядка 230 км берега вдоль Тамани находятся под угрозой абразии и оползней \cite{Misirov2024}.

Согласно исследованиям \cite{Kosyan2017} система комплексного управления береговой зоной российского сектора Чёрного моря фрагментарна: разные ведомства отвечают за землепользование, экологию, гидротехнику; отсутствует единый координационный центр. Недостаточно развита нормативная база «береговой полосы безопасности» и режима ограничения застройки, слабо внедрён кадастр морских берегов и современные системы мониторинга \cite{Gogoberidze2024}.

Подводя итог нельзя не признать, что в настоящее время ведется активная научная работа в области определения эффективных и честных компромиссов между государством и интересантами земельных отношений морского побережья, наведение правового порядка в этой сфере.
Однако объем накопленных проблем не позволяет решить их единовременно.
Принимая во внимание обстоятельства необходимости включения в плановую работу над формированием и восстановлением береговой линии воссоединяемых с Россией территорий, переоценить важность научного изучения вопросов взаимодействия города, человека и прибрежных территорий между собой \cite{Kruglova2025}.
