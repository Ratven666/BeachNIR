% \section*{\centering СПИСОК СОКРАЩЕНИЙ И УСЛОВНЫХ ОБОЗНАЧЕНИЙ}
\begin{center}
  \textbf{СПИСОК СОКРАЩЕНИЙ И УСЛОВНЫХ ОБОЗНАЧЕНИЙ}
\end{center}
\addcontentsline{toc}{section}{СПИСОК СОКРАЩЕНИЙ И УСЛОВНЫХ ОБОЗНАЧЕНИЙ}

В представляемой работе были использованы следующие сокращения и обозначения:

\vspace{1em}
\noindent \textit{ВК РФ} -- Водный кодекс Российской Федерации
    
\noindent \textit{ГИС} -- географическая информационная система
    
\noindent \textit{ГПЗУ} -- градостроительный план земельного участка

\noindent \textit{ГрК РФ} -- Градостроительный кодекс Российской Федерации
    
\noindent \textit{ЕГФД} -- Единый государственный фонд данных о состоянии окружающей среды
    
\noindent \textit{ЗК РФ} -- Земельный кодекс Российской Федерации

\noindent \textit{КоАП РФ} -- Кодекс Российской Федерации об административных правонарушениях

\noindent \textit{ПЗЗ} -- Правила землепользования и застройки

\noindent \textit{РФ} -- Российская Федерация
    
\noindent \textit{CVI} -- Coastal Vulnerability Index (Индекс уязвимости береговой зоны)

\noindent \textit{DEM} -- Digital Elevation Model (Цифровая модель рельефа)
    
\noindent \textit{DSAS} -- Digital Shoreline Analysis System (Система цифрового анализа береговой линии)
    
\noindent \textit{IPCC} -- Intergovernmental Panel on Climate Change (Межправительственная группа экспертов по изменению климата (МГЭИК))
    
\noindent \textit{OSM} -- OpenStreetMap

\noindent \textit{PCA} -- Principal Component Analysis (Метод главных компонент)
    
\noindent \textit{USGS} -- United States Geological Survey (Геологическая служба США)
