\newpage
% \section*{ЗАКЛЮЧЕНИЕ}
\begin{center}
  \textbf{ЗАКЛЮЧЕНИЕ}
\end{center}
\addcontentsline{toc}{section}{ЗАКЛЮЧЕНИЕ}

Выполненный в ходе 1 этапа НИР анализ предметной области исследований показал, что вопрос эффективного, экологичного и безопасного использования прибрежных морских территорий является востребованным и актуальным по всему миру.
Множество научных публикаций по данному вопросу в высококвартильных изданиях, международных конференций и исследований подтверждает тезис, что проблема разрушения и деградации прибрежных полос наблюдается по всему миру и до сих пор не имеет однозначного решения.

Проведенный анализ истории взаимодействия человека с объектом исследования позволяет утверждать, что несмотря на объективные риски освоения прибрежных земель, блага от такого соседства позволяют компенсировать затраты как на научное сопровождение работ по изучению процессов происходящих на прибрежных территориях, так и мероприятия по овеществлению сформулированных в их ходе рекомендаций.
Научный поиск позволяет также утверждать, что мероприятия организованные по защите и укреплению береговой линии в перспективе окупаются, чего нельзя сказать про бездействие в этом вопросе.

Анализ методов описания вероятности проявления негативных гидродинамических процессов вдоль берега показал, что наиболее изученным и универсальным показателем для его описания является индекс уязвимости береговой зоны (Coastal Vulnerability Index (CVI).
Международный опыт использования этого показателя доказал его применимость в вопросах территориального планирования объектов вдоль береговой линии морей и океанов.
Однако, несмотря на большое количество публикаций по развитию и применению CVI опыт его применения для прибрежных территорий Российской Федерации ограничивается всего парой исследований.
Объективным препятствием к исправлению ситуации в этом вопросе является отсутствие прямого доступа к подготовленным данным, необходимым для его расчета.

Выполненный в третьей части работы анализ открытых источников глобальных данных позволяет надеяться на возможность формулирования методики их агрегации для вычисления показателя CVI на требуемой для территориального планирования территории, усилия на что и будут направлены в рамках следующего этапа НИР.
Сформированные при этом методы получения и обработки геопространсвенных данных могут быть применимы в решении научных и инженерных задач применительно и к прочим территориям.
