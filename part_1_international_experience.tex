\subsection{Современный мировой опыт формирования и анализа \mbox{прибрежных пространств}}


Очевидно, что было бы искажением действительности ограничивать распределение проблем взаимодействия человека, города и моря только отечественной географией, так как в разной степени они свойственны для всех стран, имеющих выход к морскому и океаническому берегу.

\begin{figure}
    \begin{center}
	\includegraphics[width=\textwidth]{fig/DeltaWorks.jpeg}
	\caption{-- Проект Delta Works (Нидерланды)~\cite{deltaprogramme_southwest_delta}}
	\label{pic:DeltaWorks}
    \end{center}
\end{figure}

Для определенных территорий проблемы вырождаются в риск наводнений и цунами, пригоняемых штормами.
Так система дамб "Delta Works" (рисунок \ref{pic:DeltaWorks}) в Нидерландах является одним из крупнейших комплексов берегозащитных проектов мира \cite{Pilarczyk2012}, защищающим Роттердам и дельту Рейна–Мааса от штормовых нагонов.
Посоревноваться, в перспективе, с ней сможет проектируемая система "Texas Coastal Barrier" для защиты побережья Техаса и Хьюстона от ураганов \cite{Rasmussen2023}.
Классические берегозащитные укрепления и наращивание площадей пляжей реализуются в рамках защиты Бактонского газового терминала в Норфолке и прибрежных поселков в Английском Ланкашире (проекты "Great Sea Wall" и "Rossall \& Anchorsholme Coastal Defence Scheme").
Однако самым масштабным проектом является \enquote{Великая Японская стена} протяженностью более 400 км, строительство которой было интенсифицировано после разрушительного цунами 2011 года \cite{Wachter2023}.

Масштабные проекты по \enquote{отвоевыванию} земли у моря реализуются в Лагосе (Нигерия) "Eko Atlantic City" \cite{Wasiu2021}, индонезийской Джакарте "The Great Garuda" \cite{EkaPermanasari2019}, в программах "Land Reclamation", предполагающей крупные отсыпные проекты Гонконга и Шанхая (Chek Lap Kok, Lantau, Pudong и т.д.)\cite{Lai2019}.
Применим здесь и обширный Санкт-Петербургский опыт формирования намывных территорий.

Научный интерес многих исследователей также сконцентрирован относительно более локальных проектов.
Так в работе \cite{EscuderoCastillo2018} разбирается, как застройка на барьерном острове курорта Канкун (Мексика) привела к эрозии пляжа и утрате его защитных экосистемных свойств.
Авторы \cite{Oliveira2024BeachNourishment} методами математического моделирования исследовали влияние шторма Hercules 2014 года на эффективность береговой защиты пляжа Кошта-да-Капарика (около Лиссабона) и обосновали объемы восполнения песчаного пляжного материала.
Вторит этой идее и Джеймс Хьюстон (один из ключевых американских специалистов по берегозащите): в своей работе \cite{Houston2022BeachNourishment} он получил схожие результаты при анализе воздействия урагана Sandy.
Рассмотрев три стратегии защиты побережья: удаление застройки вглубь суши, строительство инженерных сооружений (буны, моллы, волноломы и т.п.) и мероприятия по пляжной подсыпке автор обосновал, что именно последний вариант даёт наилучшее сочетание снижения ущерба, устойчивости и окупаемости затрат на свое производство.
Оценка автора показывает, что проекты подсыпки под управлением Корпуса инженеров США позволили избежать порядка 1,3 млрд долларов ущерба прилегающей к морю инфраструктуре \cite{Houston2022BeachNourishment}.

Основываясь на данных дистанционного зондирования Земли, исследование прибрежной зоны пляжа Келананг \cite{MatIsa2023CoastalKelanang} определило связь эррозии и сокращения пляжной зоны с вырубкой мангровых деревьев вдоль берега. 
Предлагая использовать комплекс гидротехнических и \enquote{мягких} мер, включающих реозеленение пляжной зоны, авторы подчеркивают важность междисциплинарного подхода (ландшафтные архитекторы, инженеры, экологи) в решении аналогичных проблем возникающих в мире.

Ранжирование берегозащитных мер, наилучшим образом подходящих для конкретных типов берегов и условий среды выполнено в работе \cite{Sauve2022CoastalDefence}.
Обобщив опыт 411 научных исследований авторы предлагают методику \enquote{динамичного анализа условий} для выбора наиболее эффективных мер защиты.
Схожие идеи транслируются через труд \cite{Huynh2024HybridCoastalDefence}, в котором на основе анализа 304 исследований (875 наблюдений) были выделены четыре типа мер защиты:

\begin{itemize}
\item \textit{hard} -- \enquote{жесткие} меры (дамбы, волноломы и т.п.);
\item \textit{soft} -- \enquote{мягкие} меры (подсыпка, искусственные пляжи);
\item \textit{natural} -- \enquote{естественные} меры (восстановление экосистем, естественные мангровые леса, рифы и т.д.);
\item \textit{hybrid} -- \enquote{гибридные} меры (сочетание сооружений и экосистем).
\end{itemize}

По результатам анализа гибридные и мягкие меры в среднем показали лучшее снижение риска, показав заметно лучшие результаты относительно естественных \enquote{голых} (unvegetated) берегов.
Все рассмотренные варианты берегозащиты предполагающие человеческое участие (hard, soft, hybrid) имеют положительную экономическую отдачу в перспективе 20 лет, но мягкие и гибридные меры оказываются более рентабельны, чем чисто жёсткие сооружения.

Подтверждают эти выводы и результаты опубликованные в \cite{pinto2023coastal_defence_monitoring}.
В рамках анализа 355 кейсов из 301 научных публикаций авторы отдают предпочтение «мягким» мерам берегозащиты (подсыпка пляжей, наращивание дюн), указывая в то же время на недостаточный контроль за глобальным мониторингом прибрежной морской зоны (меньше 5\%) для контроля траектории изменения побережий в условиях происходящих климатических изменений.

Несмотря на известность мер, общие мировые тренды констатируют печальный факт того, что в борьбе за сохранение береговых линий человек проигрывает силам природы.
Так в статье \cite{Murthy2022CoastalResearch}, авторами, представляющими Министерство наук о Земле Индии, отмечается, что значительная часть пляжей в мире и на индийском побережье испытывает хроническую долгосрочную эрозию, что ведет к потере территории и деградации прибрежных экосистем.
Причину этого авторы определяют в усиливающимся антропогенном воздействии в прибрежных зонах и ускоренном повышения уровня моря вследствии глобального потепления.
Солидарны с такими выводами и авторы \cite{angnuureng_challenges_2025}, отмечающие рост масштабов береговой эрозии во всём мире и постепенный отход от «жёстких» инженерных сооружений в пользу природо-ориентированных решений.
  
Оригинальный подход к берегозащите предлагают авторы \cite{Chen2022GreenNourishment}, рассматривающие комплекс пляжной подсыпки в совокупности с высадкой морских водорослей и растений, предполагая усиление эффекта погашения волновой энергии и удержание наносов.
Моделирование показало, что наибольший эффект достигается в случае, если луг расположен в зоне прибойного вала, а намыв, в свою очередь, «экранирует» траву от разрушающего воздействия волн.
В результате предложенный метод "green nourishment" (\enquote{зелёная подсыпка}) рассматривается как перспективное \enquote{естественное} ("nature-based") решение для защиты низких песчаных берегов от эрозии и затопления, способное уменьшать штормовую опасность и одновременно поддерживать морские экосистемы.

Эффективность совмещения классической пляжной отсыпки с периодическим перераспределением пляжного материала по площади с применением бульдозеров и грейдеров предлагается в статье \cite{PELLON2023}. 
Авторы акцентируют внимание на то, что естественное накопление осадков в спокойные периоды уже не компенсирует зимнюю штормовую эрозию, вследствие чего нужны дополнительные меры против прогрессирующего размыва берега.
Рассматривают ширину пляжа как ключевой параметр для береговой защиты, туризма и других экосистемных услуг пляжей предлагаемая авторами технология "beach scraping" продемонстрировала эффективность как в лабораторных, так и в натурных условиях на примере пляжа Фуэнтебравия (Кадис, Испания).

Основываясь на рассмотренных источниках, можно утверждать, что вопросы мониторинга динамики береговой линии и управлением ее состоянием являются важными, международно востребованными и актуальными.
Мировой опыт доказывает, что пренебрежение ими создает предпосылки для аварийных ситуаций, цена ликвидации последствий которых в динамике превышает затраты на мероприятия по их предотвращению.
