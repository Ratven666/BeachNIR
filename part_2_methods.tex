
\newpage
\section{Методология решения проблемы}

\subsection{Индекс уязвимости береговой зоны (CVI)}\label{sec:cvi}

Coastal Vulnerability Index (CVI) (Индекс уязвимости береговой зоны) -- один из наиболее распространенных и методологически выверенных подходов к количественной оценке уязвимости морских побережий. 
История его возникновения неразрывно связана с ростом научного интереса к проблемам изменения климата и повышения уровня моря, произошедшим в начале 1990-х годов. 
К настоящему времени CVI прошел путь от экспериментальной методики NASA до глобального стандарта в управлении прибрежными зонами.

Основоположником метода определения CVI считается Вивьен Горниц (Vi\-vien Gornitz) и ее коллеги из Института космических исследований Годдарда (NASA Goddard Institute for Space Studies). 
В 1990–1991 годах на фоне первых докладов IPCC и растущей обеспокоенности глобальным потеплением, перед учеными встала задача создать инструмент для оценки риска затопления и эрозии побережья США. 
В рамках решения этой задачи Горниц предложила алгоритм, который объединял разрозненные физические параметры в единый числовой показатель уязвимости.
Первая версия CVI была протестирована на Восточном побережье США и заложила концептуальную основу для всех будущих модификаций. \cite{go02000x}.

В конце 1990-х годов Геологическая служба США (USGS) инициировала Национальную оценку уязвимости побережья, в ходе которой Роберт Тилер (Robert Thieler) и Эрика Хаммар-Клозе (Erika Hammar-Klose) доработали методику Горниц, создав то, что сегодня считается «классическим CVI» \cite{ThielerHammarKlose1999}.

Они зафиксировали список из 6 ключевых факторов, определяющих вероятность возникновения негативных деформационных процессов, и выразили формулу расчета \eqref{eq:CVI}, которая используется в большинстве исследований до нашего времени:

\begin{equation}\label{eq:CVI}
CVI= \sqrt{\frac{a \cdot b \cdot c \cdot d \cdot e \cdot f}{6}},
\end{equation}

\noindent где переменные $a$–$f$ соответствуют ранжированным по баллам (от 1 (очень низкая) до 5 (очень высокая)) оценкам для факторов:

\begin{itemize}
\item $a$ -- геоморфология берега (скалы — 1, песчаный пляж — 5), определяет естественную устойчивость берега;
\item $b$ -- скорость изменения береговой линии (м/год, из DSAS-анализа), характеризует тренды эрозии и аккреции берега;
\item $c$ -- уклон берега (крутой — низкая уязвимость, пологий — высокая), определяет потенциал к затоплению;
\item $d$ -- скорость относительного повышения уровня моря (мм/год), определяющая глобальные климатические и тектонические изменения;
\item $e$ -- средняя значительная высота волны (м), определяющая воздействие волновой энергии на берег;
\item $f$ -- средняя величина прилива (м), характеризующая приливно-отливную динамику.
\end{itemize}

Полученные значения CVI разбиваются на квартили для определения уровня уязвимости прибрежной территории (низкая, умеренная, высокая, очень высокая) \cite{Tarigan2024CVI}.

По мере распространения и интеграции ГИС-технологий и формирования концепций устойчивого развития урбанизируемых территорий, методика CVI начала трансформироваться и дополняться факторами, расширяя формулу \eqref{eq:CVI} по аналогии с уже введенными параметрами.
В ответ на распространенную критику классического CVI за игнорирование человеческого фактора появились модифицированные индексы,  которые начали включать плотность населения, наличие инфраструктуры и экономическую ценность земель \cite{McLaughlinCooper2010, SzlafszteinSterr2007}.
Выведенный в 2025 году \enquote{геотехнический} Geotechnical Coastal Vulnerability Index (GCVI), включил в себя свойства грунтов и гранулометрический состав наносов, показав свою эффективность на примере Патрасского залива в Греции \cite{Boumpoulis2025GCVI}.

Уточнение индекса под конкретные локальные территории сформировало отдельные региональные модификации в виде индексов территориальной уязвимости побережья  Coastal Territorial Vulnerability Index (CTVI).
В работе \cite{Barros2022CTVI} была предложена методология адаптированная под условия трех районов Португалии, по которым имелись исторические данные о динамике прибрежных территорий. 
Предложенный CTVI включал 33 переменные, сгруппированные в четыре группы:

\begin{itemize}
\item морфологический компонент ($Mv$) -- 6 переменных​ (тип берега, уклон, ширина пляжа, защита береговой линии, геология пород, характеристики прибрежных дюн);
\item стоимость земли ($Lv$) -- 2 переменные (Тип землепользования, коэффициенты налога на имущество);
\item здания ($Bv$) -- 10 переменных: (конструкционные материалы, этажность, состояние объекта, наличие подземных этажей, ориентация относительно побережья, гидродинамические характеристики первого этажа, тип использования, число жителей (посетителей), наличие критически важных элементов);
\item общественные пространства ($PAv$) -- 15 переменных (число пользователей, временная динамика занятости, подвижные объекты, критическая инфраструктура, садово-парковое оборудование и пр.).
\end{itemize}

На основе многокритериального анализа были вычислены веса для каждой переменной после чего CTVI рассчитывается как сумма по четырем категориям \eqref{eq:CTVI}:

\begin{equation}\label{eq:CTVI}
CTVI = Mv + Lv + Bv + PAv,
\end{equation}

\noindent где каждый компонент нормализован в диапазоне от 0 до 1.

Итоговый CTVI принимает значения от 0 (минимальная уязвимость) до 4 (максимальная уязвимость). 
Результаты расчета валидировались на основе исторических данных о  650 случаях прибрежного затопления и переливов за период 1980-2018 годов, при этом 83,3\% случаев произошли в зонах, классифицированных как имеющие умеренную, высокую и очень высокую уязвимость по CTVI.

В целом, наиболее эффективные методы оценки в той или иной степени используют внутреннее ранжирование факторов, используя как субъективные (экспертные), так и объективные (статистические) подходы для определения весовых коэффициентов.
В упомянутой ранее работе \cite{Boumpoulis2025GCVI} для этого использовался метод главных компонент (PCA), но наиболее перспективным считается энтропийный метод, основанный на вариабельности данных в котором параметры с высокой информационной полезностью (низкая энтропия) получают больший вес.
Так в работе  \cite{Fu2022HainanErosionCVI} были объединены метод Дельфи и энтропийный метод для оценки коралловых рифов Хайнаня (Китай) в результате динамические параметры (скорость эрозии береговой линии, изменение изобат) получили наивысшие веса, что было подтверждено корреляционным анализом.

Анализ вариативности параметров в расчете может быть учтен при стохастическом подходе Probabilistic Coastal Vulnerability Index (PCVI).
Так в работе \cite{Tanim2023PCVI} на анализе совместных распределений биофизических и социально-экономических факторов уязвимости было выполнено сравнение с традиционными детерминированными аддитивным и мультипликативным индексами (ACVI и MCVI) на примере приморских округов Южной Каролины. 
Результаты работы показывают, что PCVI лучше сохраняет многомерную информацию об уязвимости и точнее объясняет наблюдавшиеся последствия ураганов Florence (2018) и Matthew (2016) по данным послештормовых карт затопления и стоимости ликвидаций их последствий.

Итогом обзора CVI может являться исследование \cite{ElKotby2026CVIReview}, представляющее собой обзор 35 работ по применению индекса прибрежной уязвимости (CVI) для оценки рисков прибрежных зон, в котором автор систематизирует и анализирует преимущества и ограничения индексов CVI, а также показывает переход от эмпирических и ГИС-подходов к AI-ориентированным и климат-адаптивным схемам с применением методов машинного обучения для повышения объективности и прогностической точности. 
Ключевой вывод работы состоит в том, что «универсального» оптимального метода оценки уязвимости побережья не существует: пригодность той или иной схемы CVI определяется доступностью данных, а также конкретным контекстом применения индекса.

\subsubsection{Применение CVI  для прибрежных территорий \mbox{Российской Федерации}}

Несмотря на известность и изученность метода оценки прибрежных территорий по риску их уязвимости, выполнение расчетов для территории РФ носит локальный и фрагментарный характер.
На сегодняшний день не существует единого источника информации, объединяющего в себе как результаты расчета CVI так и исходных данных для его нахождения.
Принципиальной проблемой использования результатов локальных расчетов является то, что выполняемая классификация результатов проводится только относительно имеющегося набора значений, отчего агрегирование между различными дискретными работами невозможно в силу различной вариабельности рассчитанного индекса.

Ограниченное количество отечественных работ позволяет привести здесь фактически полный корпус источников.

Одна из наиболее показательных работ, описывающих расчет и анализ CVI для территории Куршской косы в Калининградской области, позволила локализовать наиболее опасные участки берега \cite{sukmanova2023cvi}.
Полученные в работе результаты (рисунок \ref{pic:CVI_Калининград}) показали корреляцию с известными локациями негативных гидромеханических процессов, локализующихся в районе г. Светлогорск и г. Балтийск  (рисунок \ref{pic:Динамика_берега_Калиниград}) \cite{Ryabchuk2021RussianBalticCoasts}. 

\begin{figure}
    \begin{center}
	\includegraphics[width=150mm]{CVI_Калининград.png}
	\caption{-- Степень уязвимости (методом CVI) побережья Калининградской области \cite{sukmanova2023cvi}}
	\label{pic:CVI_Калининград}
    \end{center}
\end{figure}

\begin{figure}
    \begin{center}
	\includegraphics[width=150mm]{Динамика_берега_Калиниград.png}
	\caption{-- Схема динамики побережья Самбийского полуострова в Калининградской области \cite{Ryabchuk2021RussianBalticCoasts}}
	\label{pic:Динамика_берега_Калиниград}
    \end{center}
\end{figure}

Анализ прибрежных территорий Финского залива был выполнен в работе \cite{Kovaleva2022EGoF_CVI}, результаты расчетов которой представлены на рисунке \ref{pic:CVI_Балтийский_залив}.
Экспертное взвешивание параметров, выполненное в работе показало ключевую роль двух параметров — прибрежной геологии и частоты штормов, а также значимость степени изрезанности/закрытости береговой линии -- именно они в наибольшей степени определяют запуск и развитие абразионных процессов.
В выводах работы предлагается использовать полученные результаты при планировании строительства портов, зон рекреации и других форм освоения территории, а также выделения приоритетных зон, где необходимы меры гидротехничекой защиты и ограничение техногенной нагрузки.

\begin{figure}
    \begin{center}
	\includegraphics[width=150mm]{CVI_Балтийский_залив.png}
	\caption{-- Взвешенный CVI для прибрежных территорий Финского залива \cite{Kovaleva2022EGoF_CVI}}
	\label{pic:CVI_Балтийский_залив}
    \end{center}
\end{figure}


Анализ территорий Невской губы также был выполнен в рамках диссертационного исследования \cite{Lednova2021NevskayaGuba}, в котором проведена оценка пространственно-временного распределения воздействия на экосистему, с особым акцентом на район аванпорта «Бронка» и прилегающую акваторию.

Для других водные бассейны РФ (Черное море \cite{Gogoberidze2022CriterionStatistical}, Каспийское море, Арктика \cite{Gogoberidze2025MurmanskRisks}, моря Дальнего Востока) CVI чаще фигурирует как методический референс, чем как самостоятельный, детально опубликованный расчет.

Расчет CVI для прибрежной зоны Черного моря был выполнен в работе \cite{Tatui2019BlackSeaErosion} с дискретизацией в 1 км.
По результатам опубликованной работы четверть всего побережья Черного моря отнесена к наиболее чувствительным  и опасным к эрозии и затоплению участкам, в основном это низменные песчаные берега и области дельт втекающих рек.
Для этих зон авторами были предложены приоритетные управленческие меры по ограничению застройки, адаптация инфраструктуры, целенаправленные берегозащитные мероприятия и усиленный мониторинг.
Несмотря на качество выполненной работы мелкий масштаб приведенных в работе иллюстраций (рисунок \ref{pic:CVI_Черное_море}) не позволяют локализовать конкретное положение опасных участков, а приведенное на иллюстрациях распределение пропорций зон не соответствует конституционному положению границ РФ, а также отсутствуют расчеты для берегов Азовского моря.
Все это позволяет лишь проиллюстрировать и отметить фактическое наличие проблемных зон на прибрежной территории Краснодарского края, республики Крым и Херсонской области.

\begin{figure}
    \begin{center}
	\includegraphics[width=150mm]{CVI_Черное_море.png}
	\caption{-- Результаты расчета CVI для Черного моря~\cite{Tatui2019BlackSeaErosion}}
	\label{pic:CVI_Черное_море}
    \end{center}
\end{figure}

\subsubsection{Выводы}

Давая характеристику CVI, как показателю позволяющему оценить устойчивость береговой линии, следует признать его эффективность и удобство в качестве метрики, характеризующей вероятность проявления на территории деструктивных гидромеханических процессов.

В то же время анализ работ по его использованию для морских прибрежных территорий РФ показал ограниченное использование как в научной, нормативной так и методической литературе.
Имеющийся опыт применения (относящийся к прибрежным территориям Черного и Балтийского морей) не охватывает всю географию нашей страны, а несогласованность показателей между собой не позволяет выполнить объединение их результатов в единой шкале.

Последнее утверждение в целом относится ко всему корпусу работ по этой теме.
Естественная простота и наглядность формулы \eqref{eq:CVI}, по которой вычисляется CVI позволяет легко ее модифицировать путем добавления или исключения новых признаков, а использование ранжированных показателей -- игнорировать проблемы их размерностей и вариации.
Цена такой свободы проявила себя в том, что фактически каждое отдельное исследование находится в своей собственной, уникальной шкале, используемой только внутри отдельной, рассматриваемой территории.
В связи с этим, зоны, принимаемые как умеренно опасными относительно своих соседей, были бы безопасными в других местах, обладающих большей динамикой изменения береговой линии.
Снизить этот эффект можно в ходе совместного анализа всей прибрежной территории, что, очевидно, потребует формирования единой методики сбора и обработки пространственных данных. 

\subsection{Идея нового предлагаемого в НИР подхода}

Основываясь на выводах предыдущего раздела, можно сформулировать следующую генеральную задачу -- разработка единого метода для формирования возможности вычисления CVI одновременно на всей территории РФ.
Условием для существования этого метода является возможность получения полного набора необходимых параметров для вычисления CVI в любой точке побережья, что может быть достигнуто путем формирования единой базы данных их содержащих или стабильных методов их получения и расчета в режиме реального времени.
Опыт проанализированных работ показал, что часть необходимых параметров может быть рассчитана на основе данных находящихся в открытом доступе или доступном в режиме ограниченного доступа.
Проблемой при этом остается их разрозненность и разнородность, требующая от исследователя больших затрат времени и сил на их агрегацию и предварительную обработку.

Разработка единого интерфейса совмещающего инструменты получения, обработки и вычисления результирующего CVI позволили бы стандартизировать эти процедуры.
Очевидным недостатком такого подхода, в первое время, стала бы ограниченность ряда используемых параметров доступных к обработке, но наличие единого \enquote{фундамента} позволило бы перенаправить высвободившиеся силы исследователей на согласованное улучшение формируемой базы делающего работы в этой области более последовательными, надежными  и верифицируемыми.

