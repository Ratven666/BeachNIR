
\newpage
\section{Основные понятия, сущности и определения}

\epigraph{
    Если выпало в~Империи родиться,\\
    лучше жить в~глухой провинции у~моря.
}{\textit{«Письма римскому другу»}\\И.~Бродский}


Невозможно строго доказать, но, наверное, каждому будет легко внутри себя согласиться с тем, что стремление человека к воде является естественным.
С физической точки зрения вода -- это второй после воздуха элемент, без которого невозможно существование высокоразвитых живых организмов.
С метафизической точки зрения водные объекты -- неиссякаемый источник внутреннего вдохновения и отдохновения, наполняющий человека спокойствием и гармонией как по отношению к себе самому, так и окружающим.
С экономической, именно близость к водным объектам позволила людям раскрыть свой потенциал, на всем протяжении истории открывая перед смотрящим новые горизонты своего использования.

Уникальность береговой линии, как  главного элемента, формирующего вокруг себя прибрежные территории, состоит в том, что она является местом пересечения основных природных сил: воды, земли и воздуха. Добавив к ним (для поэтической полноты) силы огня, проявляющиеся в форме теплоты солнечного света, мы получим практически сакральное место в быту, носящее скромное имя: \enquote{пляж}. 
Переведя это на сухой язык науки, объект исследований можно определить как место, формируемое одновременным проявлением гидродинамических, геомеханических, аэрологических и термодинамических процессов, определяющих необходимость междисциплинарных знаний для своего изучения и управления.

Необозримый промежуток времени проведенного людьми в таком соседстве не мог не сформировать внутренние противоречия и вызовы, таящие в себе наравне с очевидными благами специфичные опасности, реагировать на которые было вынуждено человеческое общество. Глубина необходимой, как, впрочем, и доступной, для человека рефлексии происходящих вокруг него природных и антропогенных процессов до сих пор не позволяет в полной мере разрешить полный корпус противоречий, формирующихся внутри такого важного, интересного и ценного ресурса как прибрежные территории.


В ограниченных рамках представляемой работы невозможно в полной мере отразить богатую историю взаимодействия человека с прибрежными территориями, отчего дальнейшее конспективное изложение преследует своей целью лишь сфокусировать интерес читателя к важности изучения данного вопроса. 
Сложность явлений происходящих вокруг причалов, набережных и пляжей определяет актуальность и необходимость данной работы в контексте гармоничного развития урбанизированных прибрежных территорий.  
