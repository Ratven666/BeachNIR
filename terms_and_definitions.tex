\newpage
% \section*{ТЕРМИНЫ И ОПРЕДЕЛЕНИЯ}
\begin{center}
  \textbf{ТЕРМИНЫ И ОПРЕДЕЛЕНИЯ}
\end{center}

\addcontentsline{toc}{section}{ТЕРМИНЫ И ОПРЕДЕЛЕНИЯ}

\vspace{1em}
\noindent \textit{Абразия} (от лат. abrasio -- «соскабливание, сбривание») -- это геоморфологический процесс механического разрушения берегов водоёмов (морей, океанов, озёр, водохранилищ) волнами, течениями и прибоем

\noindent \textit{Берег} -- полоса суши, на которой имеются формы рельефа и накопления наносов, созданные волнением при современном среднемноголетнем уровне воды \cite{sp_277_1325800_2016}

\noindent \textit{Береговая зона} -- зона, состоящая из трех геоморфологических элементов: берега, подводного склона и пляжа  \cite{sp_277_1325800_2016}

\noindent \textit{Береговая линия} -- среднемноголетнее положение уреза воды \cite{sp_277_1325800_2016}

\noindent \textit{Берегозащитное (берегоукрепительное) сооружение} -- гидротехническое сооружение для защиты берега от размыва и разрушения \cite{gost_r_59241_2020, sp_277_1325800_2016}

\noindent \textit{Буна} -- пляжеудерживающее сооружение для удержания наносов из естественного вдольберегового потока наносов и сохранения естественного или искусственного пляжа в межбунных отсеках \cite{sp_32_103_97}

\noindent \textit{Вдольбереговое перемещение наносов} -- явление массового однонаправленного перемещения наносов вдоль берега, называемое также их продольным перемещением, которое происходит при подходе волн под острым углом к берегу и обуславливается наличием вдольбереговой составляющей потока волновой энергии или под воздействием течений неволновой природы \cite{sp_32_103_97}

\noindent \textit{Вдольбереговой поток наносов} -- однонаправленное результирующее перемещение наносов вдоль берега за длительный интервал времени \cite{sp_277_1325800_2016}

\noindent \textit{Ветровые течения} -- течения на водной поверхности, создаваемые касательными напряжениями, вызванными действием ветра \cite{sp_277_1325800_2016}

\noindent \textit{Динамика береговой зоны} -- совокупность береговых процессов по перестройке берега и подводного берегового склона \cite{sp_277_1325800_2016}

\noindent \textit{Искусственный пляж} -- пляж, созданный при участии антропогенных \\ средств доставки наносов в береговую зону (относится к гидротехническим сооружениям, может использоваться как в берегозащитных, так и рекреационных целях) \cite{sp_277_1325800_2016}

\noindent \textit{Пляж} -- форма рельефа береговой зоны (природная или искусственно созданная), сложенная наносами, образованная в зоне действия прибойного потока \cite{gost_r_59241_2020}

\noindent \textit{Пляжный материал} -- совокупность несцементированных обломочных осадков различной зернистости (песок, гравий, галька), аккумулирующихся в зоне берегового вала и подводного профиля пляжа под воздействием волновых, течений и литодинамических процессов

\noindent \textit{Пляжная подсыпка} (англ. beach nourishment, beach replenishment, beach fill) -- это искусственное размещение пляжного материала в прибрежной зоне для восполнения потерь наносов, вызванных естественной эрозией и абразией

\noindent \textit{Размыв} -- перемещение грунтов морского дна, пляжа, элементов наброски или крепления откоса от воздействия волн и течений \cite{gost_r_59241_2020}

