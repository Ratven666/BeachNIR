\subsection{Основные понятия, сущности и базовые процессы}

Обобщая изложенный ранее материал, следует признать отсутствие конкретного определения, формализирующего такое комплексное понятие как морские прибрежные территории.
Используемые в ВК РФ понятия \enquote{водоохранных зон} и \enquote{прибрежных защитных полос} преследуют цель определить пространство минимизирующее риск для самого водного объекта.
Определения из гидротехники позволяют описать береговую зону уже как комплекс состоящий из берега, подводного склона и пляжа, замкнув механические процессы происходящие между ними в одну систему.
Однако ни один из этих подходов, решая собственные, профильные задачи, не учитывает явно сценарии использования территорий прилегающих к берегу тем самым вырывая их из окружающего контекста.
Таким образом, комплексность объекта исследования требует для формулирования своего определения своей декомпозиции по основным особенностям.

Самым простым и очевидным признаком выделяющим прибрежные зоны является близость к воде, не самая глубокая мысль, но она сразу определяет несколько важных условий.
Наличие специфичного признака в расположении территории сразу ограничивает существующие площади суши подпадающие под формулируемое определение.
Следующий эпитет \enquote{морские} еще более сужает область исследования.
Исторический раздел представляемой работы позволяет утверждать, что прибрежные земли являются не только ограниченным, но и ценным ресурсом.
Специфичные качества прибрежных территорий делают их восстребованными в различных отраслях народного хозяйства, начиная от промышленного (портового) и заканчивая рекреационным и жилым назначением.
Важно, что в случае последних на область исследования накладываются еще ограничения по климату, который мог бы показаться привлекательным для приезжающих и рельефу местности, определяющему привлекательность и экономическую эффективность участков местности под жилое и инфраструктурное строительство.

В то же время противоречия в интересах землепользователей прибрежных территорий формируют неразрешимые конфликты интересов. 
При этом точкой фокуса, градус конфликта интересов в которой достигает максимального накала, являются прибрежные города.
Равновесие между сосредоточенными в них промышленными предприятиями и исторической застройкой в настоящее время нарушается возрастающими на эти территории объемами туристической нагрузки и дополнительного жилого строительства, продуцирующего больше инвестиционные и товарные функции земель.

Формируемые при ускоряющейся урбанизации прибрежных территорий процессы запускают некую автокаталитическую реакцию -- привлекательные земли становятся дефицитнее, цена их возрастает, инвестиционная привлекательность возрастает, позволяя переводит в класс привлекательных участки ранее не рассматриваемые как рентабельные.
Происходящие при этом увеличение антропогенной нагрузки на земли выводит их из баланса природного равновесия и запускает механизмы разрушения.

Основные законы РФ согласованно определяют необходимость мероприятий по поддержанию и сохранению экологических и топографических свойств всех территорий нашей Родины.
Сохранение этих свойств -- залог стабильности материального благополучия, как нашего, так последующих поколений.
В то же время, это благополучие недостижимо без разумного использования доставшихся нам территорий, что в свою очередь выводит их из природного равновесия.
При этом поиск сиюминутного соблазна экономической выгоды в моменте часто приводит к игнорированию перспективных гидромеханических и геомеханических опасностей, оставляя их разрешение будущим землепользователям.

Специфичной чертой морских прибрежных территорий является то, что процессы деградации и разрушения этих земель происходят кратно интенсивнее чем у других водных объектов.
Происходящий вследствие абразии и переноса наносов размыв берега представляет непосредственную угрозу разрушения для прилегающих к береговой линии объектов и потенциальную опасность для людей, на них находящихся.
Скорость протекания этих деструктивных процессов в основном определяется геологическим строением береговой линии и интенсивностью волновых и штормовых нагрузок. 
При этом интенсивность возникающих волновых процессов напрямую связана с доступной к волновому нагону площади водного объекта, что очевидно максимально проявляет себя на океанских и морских побережьях, что требует для своего изучения привлечения отдельных метеорологических и топографических знаний. 

Таким образом несмотря на разные взгляды на способы использования прибрежных земель, необходимость контроля и поддержания их стабильного состояния сформировала целый класс гидротехнических мероприятий по берегозащите.
Форма и назначение берегозащитных сооружений может быть различной и допускать различные сценарии их использования.
Однако разнообразие возможных мер защиты не позволяет сформировать универсального способа, отвечающего одновременно интересам всех сторон.

Фактически все берегозащитные сооружения определяет, помимо основной функции, пугающе высокая стоимость работ по их сооружению и реконструкции, зачастую недоступная к оперативному выделению в случае черезвычайной ситуации.
Экономически обоснованное разрешение таких ситуаций может быть достигнуто только при плановой и централизованной работе по планированию и проектированию объектов на прибрежных территориях, согласованной с оценкой потенциальных рисков в конкретной локации и систематическому мониторингу береговой линии.
Все эти работы сами по себе требуют выделения существенных человеческих и материальных ресурсов, рациональность привлечения которых может быть достигнута путем классификации прибрежных зон по степени опасности проявления негативных явлений.

\subsubsection{Определение \enquote{морской прибрежной территории}}

В связи с отсутствием утверженного в нормативных документах или общеиспользуемой практике термина \enquote{морские прибрежные территории} его необходимо сформулировать в рамках выполняемой работы.
Исходя из описанных ранее условий, под этим понятием в рамках выполняемой НИР будет пониматься следующие определения:

\vspace{1em}
\noindent \textit{Прибрежная территория} -- это часть земной поверхности, граничащая с водным объектом и характеризующаяся взаимным влиянием суши и водного объекта на состояние друг друга.

\vspace{1em}
\noindent \textit{Морская прибрежная территория} -- это прибрежная территория, граничащим водным объектом которой является море или океан.

\vspace{1em}
\noindent \textit{Урбанизированная прибрежная территория} -- это прибрежная терртория, находящаяся в границах населенного пункта, объекты которой состоят в технической, экономической, социально-культурной связи с водным объектом.


\subsubsection{Диаграмма сущностей}

Обобщая сказанное, морские прибрежные территории являются лакомым и желанным объектом, формирующим вокруг себя клубок противоречий интересов множества сторон.
Попытка распутать и препарировать который представлена в схеме, представленной на рисунке 
\ref{pic:ДиаграммаCущностей}.

Базовым объектом в представленной диаграмме является \enquote{Объект} -- это абстрактная сущность обладающая несколькими необходимыми и имманентными в своих реализациях атрибутами: состоянием, стоимостью, пользователем, назначением и степенью потенциальной опасности (риском).
В конкретной реализации объект может являться контейнером для других объектов, становясь при этом комплексным.
Базовые атрибуты объекта в целом говорят сами за себя, но понятие \enquote{Пользователя} следует рассмотреть подробнее.
Одним из методов \enquote{Объекта} является общий метод \enquote{приносить пользу}, реализуемый в потомках объекта за счет полиморфизма, выгодоприобретателем при этом становится \enquote{Пользователь}, аккумулирующий ее в себе в виде условных \enquote{денег}, имеющий возможность ими распоряжаться, но и нести ответственность за относящиеся к нему объекты.
Специфичные способы использования реализуются через спецификацию \enquote{Назначения} объекта, но универсальными являются процессы разрушения объекта (снижения собственного состояния), методы контроля состояния и методы его восстановления, требующие для своего выполнения времени и денег, что обеспечивается пользователем объекта.
Стремление к сбалансированной минимизации этих параметров достигается за счет возможности использования специфичных методов контроля и восстановления реализуемых для каждого типа объектов индивидуально.

% \begin{landscape}
% \begin{figure}[!htbp]
%     \centering
%     \makebox[0pt][c]{
%         \includegraphics[width=1.15\paperwidth,keepaspectratio]{fig/Диаграмма_сущностей}
%     }
%     \caption{Диаграмма сущностей системы}
%     \label{pic:ДиаграммаCущностей}
% \end{figure}
% \end{landscape}

Рассматриваемая на схеме сущность \enquote{Территория} является реализацией класса \enquote{Объект}, дополняя его собственными специфичными характеристиками (рельефом, климатом, запасами полезных ископаемых и т.п.) и отдельно выделяемым \enquote{Типом территории}.
Имплементируемый \enquote{Тип территории} позволяет стандартизировать территории по комплексу из двух признаков: типу \enquote{Локации} и \enquote{Способу Защиты} территории, необходимость которого следует по закону.
Сущность \enquote{Локации} является абстрактной, но позволяет классифицировать территории по собственным специфичным признакам.
Так в рассматриваемом примере они разделены на \enquote{Прибрежные} и \enquote{Материковые}.
Ключевой особенностью \enquote{Прибрежного} типа является динамичность береговой линии, механизм протекания и скорость которой определяется принадлежностью к конкретному типу водного объекта.

Наравне с \enquote{Локацией} необходимым для \enquote{Типа Территории} является и ее \enquote{Способ Защиты}.
Для всех объектов реализующую эту сущность свойственно наличие таких атрибутов как цена (стоимость работ по организации защиты), ее эффективность (определяющая темп скорости снижения \enquote{состояния} территории и включенных в нее \enquote{объектов}), срок службы и опциональные дополнительные функции и включаемые \enquote{объекты}.
Базовым поведением сущности является \enquote{защищать территорию}, предотвращая ее износ.

Примерами реализации \enquote{Способов Защиты} в контексте выполняемой НИР являются \enquote{Гидротехнические} способы защиты, классификация которых выполнена на основе работы \cite{Huynh2024HybridCoastalDefence}.
Примером \enquote{дополнительных функций} некоторых из типов берегозащиты может являться рекреационное назначение или возможность размещения морских транспортных средств.

Раздробленность  между сущностями разнородных качеств определяет сложность правильного выбора конкретных объектов размещяемых на прибрежных территориях и способов их защиты.
Сущностью принимающей на себя сложность выбора последних ложится на предлагаемый к формированию в работе новый тип объекта \enquote{Фабрика Способов Защиты}.
Механизм выбора конкретного способа может быть реализован в нем различными способами, в зависимости от контекста решаемой задачи, но необходимым и фундаментальным условием определяющем систему принимаемых решений должен стать общий для всех способов источник \enquote{информации о смежных территориях}.
Этот источник должен агрегировать в себе не только информацию конкретной локации, доступную самой территории в ее атрибуте \enquote{характеристик}, но и прилегающим и содержащихся в них \enquote{объектах}.
Наличие такой информации позволит учесть интересы всех \enquote{пользователей} в принимаемом решении и достигнуть максимальной эффективности.

Очевидно, что глобальность необходимого агрегатора информации не позволяет сформировать его в рамках одного НИР, но ограничиваясь конкретными \enquote{морскими прибрежными территориями} возможно  описать локализацию скорость протекания деформационных на них процессов, определив тем самым наиболее перспективные к изучению и освоению территории.   


%%%%%%%%%%%%%%%%%%%%%%%%%%%%%%%%%%%%%%%%%%%%%%%%%%%%%%%%%%%%%%%%%%%

\subsubsection{Диаграмма процесса разработки генерального плана морских прибрежных территорий с учетом влияния водного объекта}

В связи с требованиями защиты прибрежных земель от разрушения, устанавливаемыми ВК РФ и ЗК РФ, организация необходимых мероприятия по их организации должна быть учтена при разработке или корректировке существующих генеральных планов морских прибрежных территорий, что является стартовой точкой диаграммы, изображенной на рисунке \ref{pic:ДиаграммаПроцесса}. 

% \begin{landscape}
% \begin{figure}[!htbp]
%     \centering
%     \makebox[0pt][c]{
%         \includegraphics[width=1.15\paperwidth,keepaspectratio]{fig/Диаграмма_процесса_НИР_1}
%     }
%     \caption{Диаграмма процесса разработки генерального плана морских прибрежных территорий с учетом влияния водного объекта}
%     \label{pic:ДиаграммаПроцесса}
% \end{figure}
% \end{landscape}

Для принятия рациональных решений при этом необходимо классифицировать береговую линию согласно вероятности проявления негативных, разрушительных процессов, вызванных влиянием моря на прибрежную территорию. 
Анализ научных источников проведенный в разделе \ref{sec:cvi} показал, что наиболее перспективным и изученным в мире параметром для описания такой вероятности является индекс уязвимости береговой зоны (Coastal Vulnerability Index (CVI)).
К сожалению, практика использования CVI в городском планировании и анализе к настоящему моменту не выполняется, хотя опыт многих стран доказывает эффективность его применения в течение последних 30 лет.

Одним из препятствий к использованию CVI является то, что для его определения необходимы специфичные исходные данные, описывающие граничащий с территорией водный объект.
Спецификация этих данных и варианты их сбора описаны в разделе \ref{sec:data} данной работы.
Определив CVI, становится возможным классифицировать участки берега по вероятности проявления на них негативных процессов разрушения береговой линии или затопления прилегающих к ней территорий.

В позитивном случае (когда вероятность наступления этих событий принимается допустимой) в работе предлагается выполнение процесса планирования территории согласно действующим методиками, с единственным дополнительным обременением, заключающимся в прописывании регламента мероприятий по мониторингу состояния береговой линии и ее восстановлению при необходимости.
Следует отметить, что предлагаемое обременение в работе предлагается использовать во всех случаях использования прибрежной территории, гарантируя тем самым контроль исполнения федеральных законов и кодексов.

Негативный сценарий анализа предполагает реакцию на наличие на рассматриваемой территории участков с высокой вероятностью негативных процессов.
В этом случае необходим дополнительный анализ исходных данных примененных в расчете CVI.
В случае сомнений в качестве использованных показателей целесообразно их уточнение путем дополнительных инженерных геодезических, батиметрических, геологических и метеорологических изысканий.
Уточнив в результате получения новых данных CVI, становится возможным точнее локализовать участки берега с неприемлемой вероятностью негативных явлений и определить границы конкретной зоны влияния водного объекта.
Ход дальнейшего процесса планирования рекомендуется выполнять с учетом типа функциональной зоны, попадающей в зону вредного влияния водного объекта.

Так, размещение на этих участках промышленных зон будет являться для города более предпочтительным, так как все мероприятия по гидрозащите лягут на плечи землепользователя.
С учетом того, что размещаемые на этих землях промышленные объекты (порты, нефтебазы, перевалочные комплексы, доки и т.п.) имеют высокий класс опасности и подпадают под контроль Ростехнадзора, мероприятия по проектированию, строительству и эксплуатации гидротехнических сооружений организуются пользователем этих объектов.
Дополнив сказанное тем, что все разрабатываемые проекты для этих объектов проходят государственную экспертизу, дополнительные решения на этапе составления генерального плана для них не требуются.

Наибольший интерес в области принятия решений по использованию прибрежных территорий остается локализован на участках жилого и рекреационного испльзования, попадающих в зону с высокой вероятностью негативных процессов вдоль берега.
В области пересечения этих зон необходим дополнительный анализ риска, выполненный с учетом специфики существующих и планируемых к размещению объектов.
В результате учет полученных показателей может являться критерием для выбора способа гидрозащиты территории, учитывающей как необходимые средства для их организации так и эксплуатации.


\begin{landscape}
\begin{figure}[!htbp]
    \centering
    \makebox[0pt][c]{
        \includegraphics[width=1.15\paperwidth,keepaspectratio]{fig/Диаграмма_сущностей}
    }
    \caption{-- Диаграмма сущностей системы}
    \label{pic:ДиаграммаCущностей}
\end{figure}
\end{landscape}

\begin{landscape}
\begin{figure}[!htbp]
    \centering
    \makebox[0pt][c]{
        \includegraphics[width=1.15\paperwidth,keepaspectratio]{fig/Диаграмма_процесса_НИР_1}
    }
    \caption{-- Диаграмма процесса разработки генерального плана морских прибрежных территорий с учетом влияния водного объекта}
    \label{pic:ДиаграммаПроцесса}
\end{figure}
\end{landscape}
