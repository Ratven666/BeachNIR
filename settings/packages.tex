% настройка кодировки, шрифтов и русского языка
\usepackage{fontspec}
\usepackage{polyglossia}

\usepackage[hyphens]{url} % улучшенные переносы в URL
\usepackage{xurl}         % разрешить перенос почти в любом месте ссылки

% рабочие ссылки в документе
\usepackage{hyperref}

% графика
\usepackage{graphicx}
\usepackage{tikz}

% поворот страницы
\usepackage{pdflscape}

% качественные листинги кода
\usepackage{minted}
\usepackage{listings}
\usepackage{lstfiracode}

% отключение копирования номеров строк из листинга, работает не во всех просмотрщиках (в Adobe Reader работает)
\usepackage{accsupp}
\newcommand\emptyaccsupp[1]{\BeginAccSupp{ActualText={}}#1\EndAccSupp{}}
\let\theHFancyVerbLine\theFancyVerbLine
\def\theFancyVerbLine{\rmfamily\tiny\emptyaccsupp{\arabic{FancyVerbLine}}}

\usepackage[
    backend=biber,
    style=gost-numeric,
    sorting=none,
    hyperref=true,
    language=auto,
    bibencoding=utf8
]{biblatex}

\renewcommand{\bibfont}{\normalfont\rmfamily\upshape}

% названия, DOI, URL без курсива — это уже работает
\DeclareFieldFormat{title}{\textup{#1}}
\DeclareFieldFormat{booktitle}{\textup{#1}}
\DeclareFieldFormat{maintitle}{\textup{#1}}
\DeclareFieldFormat{journaltitle}{\textup{#1}}
\DeclareFieldFormat{issuetitle}{\textup{#1}}
\DeclareFieldFormat{doi}{\textup{DOI:\space #1}}
\DeclareFieldFormat{url}{\textup{#1}}

% Фикс: снять курсив со всех участков, которые biblatex-gost помечает как 'emph'
\protected\def\mkbibemph#1{\textup{#1}}
% убрать курсив «шапки» записи (авторы и пр.) в biblatex-gost
\renewcommand*{\mkgostheading}[1]{#1}

% установка полей
\usepackage{geometry}

% нумерация картинок по секциям
\usepackage{chngcntr}

% дополнительные команды для таблиц
\usepackage{booktabs}
\usepackage{tabularx} 

% для заголовков
\usepackage{caption}
\usepackage{titlesec}
\usepackage[dotinlabels]{titletoc}

% разное для математики
\usepackage{amsmath, amsfonts, amssymb, amsthm, mathtools}

% водяной знак на документе, см. main.tex
\usepackage[printwatermark]{xwatermark}

\usepackage{epigraph}
\usepackage{verse}
\usepackage{appendix}
