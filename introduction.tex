\newpage
% \section*{ВВЕДЕНИЕ}
\begin{center}
  \textbf{ВВЕДЕНИЕ}
\end{center}
\addcontentsline{toc}{section}{ВВЕДЕНИЕ}

Морская береговая линия, являясь местом совокупного взаимодействия ансамбля гео-, гидро- и аэромеханических процессов, представляет собой сложную и динамическую систему, обладающую, в результате влияния внешних сил, высокой степенью потенциального риска деформаций прибрежных территорий.
Указанные деформации, в свою очередь, влекут за собой утрату конструкционной целостности находящихся в прибрежных зонах инженерных и природных объектов, что, помимо экономического ущерба, провоцирует опасность и для человека.

Эффективное управление этими процессами требует применения различных средств гидротехнических средств защиты, рациональный выбор которых определяется в ходе объемных, специфичных и высококвалифицированных работ по гидротехническому проектированию.
Примененные к проектированию прилагательные, в совокупности с ограниченными количеством квалифицированных инженеров-гидротехников, не позволяют распространить в полной мере опыт этих работ на всем протяжении береговой линии, в связи с чем они локализуются в промышленных прибрежных зонах.

При этом естественная тяга человека к водным объектам обуславливает заполнение прибрежных территорий наравне с объектами промышленного и инфраструктурного назначения рекреационными и жилыми зонами.
Помимо гуманитарной направленности, перечисленные объекты объединяет их относительно большое количество, что влечет за собой их проектирование организациями, имеющими ограниченный гидротехнический опыт.
Последствия ограниченной компетентности разработчиков и исполнителей работ в прибрежных зонах проявляются в преждевременной утрате эксплуатационных характеристик сооружаемыми объектами.
В то же время, распределение проявления негативных явлений во времени, в совокупности с неравномерностью распределения зон повышенного риска вдоль береговой линии, создают у внешнего наблюдателя иллюзию естественности и неизбежности этих процессов, выражающуюся в молчаливом принятии тезиса "вода камень точит".
Лингвистическим антагонистом этого утверждения является известное "под лежачий камень вода не течет", что формирует дугу диалектического противоречия, внести свой вклад в разрешение которого преследует, в идейном плане, цель данной работы.

Рассматривая практическую плоскость вопроса, следует опираться на тот факт, что основная масса удобных для своего освоения участков береговой линии была занята к настоящему моменту в ходе эволюционного заполнения прибрежных земель городским пространством.
Оставшиеся свободными для развития прибрежные территории требуют выполнения мероприятий по оценке своего потенциала.
Для обеспечения экономической эффективности освоения прибрежных территорий одним из факторов для такой оценки должен являться риск проявления негативных явлений, связанных с гидромеханическими процессами, накладывающими на землепользователя дополнительные траты на обустройство берегозащитных конструкций или ликвидацию последствий в случае пренебрежения этими мероприятиями.
В любом случае непреклонные силы природы возьмут свое, вопрос лишь в том, какую степень риска готовы принять ответственные землепользователь и государство, использующие свои ресурсы к достижению коллективной пользы.

Опираясь на сказанное, разработка общедоступного метода оценки прибрежных морских территорий по степени риска проявления негативных деформационных и гидромеханических процессов является актуальной задачей, решение которой повысит как экономическую эффективность мероприятий по их освоению, так и безопасность и сохранение названных земель при использовании.

\vspace{1em}
\textbf{Формулировка проблемы}

В градостроительных отношениях применяются методы формирования морских прибрежных территорий, которые не учитывают гидромеханических и геомеханических процессов вдоль морской береговой линии, что впоследствии может привести к созданию потенциально опасной ситуации для жителей города и повлечь за собой негативные экономические эффекты.


% Отсутствие доступной информации о локализации зон повышенных рисков проявления негативных гидромеханических и геомеханических процессов вдоль морской береговой линии провоцирует ошибки при проектировании и эксплуатации объектов на прибрежных территориях, что влечет за собой негативные экономические последствия для землепользователей и потенциальную опасность для жителей города.

\vspace{1em}
\textbf{Объект исследования}

Система морских прибрежных территорий, планируемая к градостроительному освоению.

% Морские прибрежные территории.

\vspace{1em}
\textbf{Предмет исследования}

Метод оценки морских прибрежных территорий по степени риска проявлений негативных гидромеханических и геомеханических процессов.

\vspace{1em}
\textbf{Цель исследования}

Разработка и внедрение метода оценки морских прибрежных территорий по степени риска проявлений негативных гидромеханических и геомеханических процессов.

% Оценка морских прибрежных территорий по степени риска проявлений негативных гидромеханических и геомеханических процессов.

\vspace{1em}
\textbf{Идея работы}

Оценка риска негативных гидромеханических и геомеханических процессов за счет экстраполяции элементов гидротехнического опыта анализа нагрузок на береговую линию и уточнение за счет вычисляемого индекса уязвимости береговой зоны (CVI).

\vspace{1em}
\textbf{Задачи исследования}

\begin{enumerate}
    \item Анализ предметной области, научный поиск и структурирование материала по предмету исследований.
    \item Определение метрики (индекса) для описания уязвимости морской прибрежной территории.
    \item Проведение анализа состава и доступности необходимых исходных данных.
    \item Анализ методологии сбора и обработки геопространственных данных для вычисления этого индекса.
    
    % \item Установление критерия, позволяющего классифицировать прибрежные территории по степени риска;
    % \item Вычисление предложенного индекса для морских прибрежных территорий РФ;
    % \item Апробация полученных результатов;
    % \item Разработка рекомендаций по формированию морской прибрежной территории с учетом риска проявления негативных гидромеханических и геомеханических событий.
\end{enumerate}

\vspace{1em}
\textbf{Методология и методы}

\begin{enumerate}
    \item Методы научного поиска и анализа.
    \item Методы сбора и обработки геопространственных данных.
    \item Методы математической статистики.
    \item Методы геопространственного анализа.
\end{enumerate}

\vspace{1em}
\textbf{Новизна исследования}

Разработанные в ходе исследования методики получения геопространственных данных и вычисленные для морских прибрежных территорий индексы уязвимости позволят повысить качество научного описания объекта исследований, формируя при этом новые знания о мире.

\vspace{1em}
\textbf{Теоретическая и практическая значимость}

\begin{enumerate}
\item Разработанные в ходе исследования методики получения геопространственных данных смогут быть использованы в смежных по тематике научных и инженерных работах;

\item Сформулированные рекомендации к процессу территориального планирования позволят учитывать специфику морских прибрежных территорий, повышая тем самым экономическую эффективность принимаемых решений. 
\end{enumerate}

\vspace{1em}
\textbf{Соответствие направлению подготовки}

Выполненный в ходе НИР анализ уязвимости морских береговых линий повысит эффективность территориального планирования, позволяя оптимизировать затраты на гидротехнические сооружения, время на подготовку проектных материалов, трудовые ресурсы, необходимые для их формирования, и сократить, в перспективе, траты на ликвидацию негативных гидромеханических и геомеханических событий, что согласуется с паспортом направления 27.04.07 «Наукоемкие технологии и экономика инноваций».
