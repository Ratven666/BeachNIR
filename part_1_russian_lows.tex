\subsection{Нормативное регулирование использования прибрежных территорий}
\subsubsection{Нормативные документы Российской Федерации}

Нормативное регулирование прибрежных территорий в Российской Федерации представляет собой сложную иерархическую систему, стартующую с Конституции РФ и доходящую до региональных нормативных актов и технических нормативов. 

Согласно 42 статье Конституции РФ: "Каждый имеет право на благоприятную окружающую среду, достоверную информацию о ее состоянии и на возмещение ущерба, причиненного его здоровью или имуществу экологическим правонарушением", что определяет необходимость своевременных защитных и мониторинговых мероприятий, обеспечивая тем самым статью 9: "Земля и другие природные ресурсы используются и охраняются в Российской Федерации как основа жизни и деятельности народов, проживающих на соответствующей территории".

Водный кодекс Российской Федерации \cite{WaterCodeRF} является базовым документом, регулирующим водные отношения. 
Ключевое значение имеет статья 65, определяющая понятия "водоохранных зон" и "прибрежных защитных полос", устанавливающая их размеры и режим использования. 
"Инженерная защита территорий и объектов от затопления, подтопления, разрушения берегов водных объектов, заболачивания и другого негативного воздействия вод" рассматривается в статье 67.1 этого документа, делегируя конкретику мероприятий по "строительству берегоукрепительных сооружений, дамб и других сооружений, предназначенных для защиты территорий...", области законодательства о градостроительной деятельности,
оставляя за государством право на изъятие земельных участков "в целях строительства сооружений инженерной защиты территорий и объектов от негативного воздействия вод" в порядке установленном земельным и гражданским законодательством.
Отдельно выделяется подпункт 7 этой статьи, определяющий обязательства по охране прибрежных территорий: "Собственник водного объекта обязан осуществлять меры по предотвращению негативного воздействия вод и ликвидации его последствий. 
Меры по предотвращению негативного воздействия вод и ликвидации его последствий в отношении водных объектов, находящихся в федеральной собственности, собственности субъектов Российской Федерации, собственности муниципальных образований, осуществляются исполнительными органами государственной власти или органами местного самоуправления в пределах их полномочий в соответствии со статьями 24 - 27 настоящего Кодекса".

Статья 6 ВК РФ регламентирует размеры береговой полосы вдоль водных объектов общего пользования.
В то время как порядок определения и оформления границ водоохранных зон и прибрежных защитных полос установлен "Постановлением Правительства РФ от 31.10.2024 № 1459" \cite{RF_Government_1459_2024}.
Важно отметить, что в обновленной редакции документа не конкретизируюется периодичность уточнения положения береговой линии, отсылая по этим вопросам к правилам, утверждённым "Постановлением Правительства РФ от 29.04.2016 № 377" \cite{postanovlenie_pravitelstva_rf_377_2016}.

Демаркировав границы водных объектов, дальнейшее правовое изучение вопроса следует выполнить со стороны суши.
Земельный кодекс РФ \cite{zemelnyj_kodeks_rf_2001} рассматривает в статье 13 вопросы охраны земель, обременяя собственников, землепользователей и арендаторов земельных участков необходимостью "защиты земель от водной и ветровой эрозии".

В Градостроительном кодексе \cite{gradostroitelnyj_kodeks_rf_2004} прямо не выделяются понятия "прибрежных территорий" и синонимичных по смыслу объектов, однако обширно используются общие конструкции про территории, на которых действуют "иные специальные режимы" (зоны с особыми условиями использования территорий, особо охраняемые природные территории, и т.д.).
Таким образом для прибрежных территорий, (согласно ГрК РФ) ПЗЗ закрепляют все водные и экологические ограничения (водоохранные и прибрежные полосы, зоны затопления/подтопления по статье 67.1 ВК РФ) и задают допустимые виды застройки и ее параметры. 
ГПЗУ затем прикладывает эти режимы к конкретному участку, указывая, что именно и на каких условиях там можно строить или размещать, превращая общие нормы ВК РФ, ЗК РФ и ГрК РФ в конкретные требования для проекта. 

Общие рамки методов контроля за состоянием берегозащитных сооружений определяет ГОСТ Р 59241-2020 «Берегозащитные сооружения. Правила обследования и мониторинга технического состояния» \cite{gost_r_59241_2020}.
Базовым документом, связывающим юридические требования по безопасности и охране среды с конкретными инженерными параметрами морских берегозащитных сооружений является СП 277.1325800.2016 «Сооружения морские берегозащитные. Правила проектирования» \cite{sp_277_1325800_2016}.
Этот документ устанавливает правила проектирования морских берегозащитных сооружений на открытых побережьях внутренних бесприливных морей и может применяться для берегов озёр и водохранилищ, задавая основу для инженерных решений по защите берега от размыва и волнения, развивая комплекс мер предусмотренных в \cite{sp_32_103_97}.
