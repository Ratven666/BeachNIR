% XeLaTeX document
\documentclass[a4paper]{article}
\usepackage[14pt]{extsizes}
\usepackage{setspace}
\onehalfspacing
\usepackage{sectsty}
\allsectionsfont{\centering}
\usepackage{amsthm}

\theoremstyle{definition}
\newtheorem{definition}{Определение}

\usepackage{enumitem}
% глобально для всех списков:
% убирает дополнительные вертикальные интервалы
% или, более явно:
\setlist{noitemsep, topsep=0pt, parsep=0pt, partopsep=0pt}
% абзацный отступ
\setlength{\parindent}{1.25cm}
\usepackage{indentfirst}
% списки с тем же отступом, что абзацы
\setlist[itemize]{leftmargin=\parindent}
\setlist[enumerate]{leftmargin=\parindent}
\setlist[description]{leftmargin=\parindent}
% глобально для всех списков
\usepackage{indentfirst}
% Нумерованные списки
\setlist[enumerate]{
  labelindent=\parindent, % номер на месте первой буквы абзаца
  leftmargin=*            % остальное enumitem подберёт сам
}
% Маркированные списки
\setlist[itemize]{
  label=--,               % тире как маркер
  labelindent=\parindent, % тире на месте первой буквы абзаца
  leftmargin=*
}

\usepackage{indentfirst}     % чтобы первый абзац после заголовка был с отступом
\setlength{\parindent}{1.25cm}

\usepackage{titlesec}
\titlespacing*{\section}{\parindent}{*1}{0.5\baselineskip}

\titlespacing*{\subsection}
  {\parindent}       % сам заголовок от левого поля с отступом
  {2\baselineskip}   % ДВЕ строки перед подзаголовком
  {0.5\baselineskip} % интервал после подзаголовка (можно подправить)

\titlespacing*{\subsubsection}
  {\parindent}              % сам заголовок от левого поля с отступом
  {1\baselineskip}   % Одна строка перед подзаголовком
  {0.5\baselineskip} % интервал после подзаголовка (можно подправить)

% 14pt для \section, \subsection, \subsubsection
\titleformat{\section}
  {\bfseries\fontsize{14pt}{14pt}\selectfont}
  {\thesection}{1em}{}
\titleformat{\subsection}
  {\bfseries\fontsize{14pt}{14pt}\selectfont}
  {\thesubsection}{1em}{}

\titleformat{\subsubsection}
  {\bfseries\fontsize{14pt}{14pt}\selectfont}
  {\thesubsubsection}{1em}{}

\usepackage[autostyle]{csquotes}
% \MakeOuterQuote{"} % опционально: чтобы "текст" → «текст»
\usepackage[russian,english]{babel}
\usepackage[autostyle]{csquotes}


% Редактируем: конфигурация, личные настройки: имя, название предмета и пр. для титульной страницы и метаданных документа здесь

\newcommand{\university}{Санкт-Петербургский горный университет императрицы Екатерины II}
\newcommand{\city}{Санкт-Петербург}

\newcommand{\faculty}{Строительный факультет}
\newcommand{\department}{Кафедра маркшейдерского дела}


\newcommand{\subject}{Основы научных исследований}

\newcommand{\workType}{Отчёт по практической работе }
\newcommand{\workNumber}{1}

\newcommand{\workTytle}{"Статистическая обработка данных"}

\newcommand{\studentGroup}{ГГ-22-1,2}
\newcommand{\studentName}{И.И. Иванов}

\newcommand{\tutorPosition}{Доцент}
\newcommand{\tutorName}{М.Г. Выстрчил}



% Не редактируем: используемые пакеты
% настройка кодировки, шрифтов и русского языка
\usepackage{fontspec}
\usepackage{polyglossia}

\usepackage[hyphens]{url} % улучшенные переносы в URL
\usepackage{xurl}         % разрешить перенос почти в любом месте ссылки

% рабочие ссылки в документе
\usepackage{hyperref}

% графика
\usepackage{graphicx}
\usepackage{tikz}

% поворот страницы
\usepackage{pdflscape}

% качественные листинги кода
\usepackage{minted}
\usepackage{listings}
\usepackage{lstfiracode}

% отключение копирования номеров строк из листинга, работает не во всех просмотрщиках (в Adobe Reader работает)
\usepackage{accsupp}
\newcommand\emptyaccsupp[1]{\BeginAccSupp{ActualText={}}#1\EndAccSupp{}}
\let\theHFancyVerbLine\theFancyVerbLine
\def\theFancyVerbLine{\rmfamily\tiny\emptyaccsupp{\arabic{FancyVerbLine}}}

\usepackage[
    backend=biber,
    style=gost-numeric,
    sorting=none,
    hyperref=true,
    language=auto,
    bibencoding=utf8
]{biblatex}

\renewcommand{\bibfont}{\normalfont\rmfamily\upshape}

% названия, DOI, URL без курсива — это уже работает
\DeclareFieldFormat{title}{\textup{#1}}
\DeclareFieldFormat{booktitle}{\textup{#1}}
\DeclareFieldFormat{maintitle}{\textup{#1}}
\DeclareFieldFormat{journaltitle}{\textup{#1}}
\DeclareFieldFormat{issuetitle}{\textup{#1}}
\DeclareFieldFormat{doi}{\textup{DOI:\space #1}}
\DeclareFieldFormat{url}{\textup{#1}}

% Фикс: снять курсив со всех участков, которые biblatex-gost помечает как 'emph'
\protected\def\mkbibemph#1{\textup{#1}}
% убрать курсив «шапки» записи (авторы и пр.) в biblatex-gost
\renewcommand*{\mkgostheading}[1]{#1}

% установка полей
\usepackage{geometry}

% нумерация картинок по секциям
\usepackage{chngcntr}

% дополнительные команды для таблиц
\usepackage{booktabs}
\usepackage{tabularx} 

% для заголовков
\usepackage{caption}
\usepackage{titlesec}
\usepackage[dotinlabels]{titletoc}

% разное для математики
\usepackage{amsmath, amsfonts, amssymb, amsthm, mathtools}

% водяной знак на документе, см. main.tex
\usepackage[printwatermark]{xwatermark}

\usepackage{epigraph}
\usepackage{verse}
\usepackage{appendix}


% Не редактируем: параметры используемых пакетов и не только
% настройки polyglossia
\setdefaultlanguage{russian}
\setotherlanguage{english}

% локализация
\addto\captionsrussian{
	\renewcommand{\figurename}{Рисунок}%
	\renewcommand{\partname}{Глава}
	\renewcommand{\contentsname}{\centerline{Содержание}}
	\renewcommand{\listingscaption}{Листинг}
}

% основной шрифт документа
% \setmainfont{CMU Serif}
% \newfontfamily\cyrillicfont{CMU Serif}[Script=Cyrillic]
% Шрифт Times New Roman
\setmainfont[Ligatures={TeX,Historic}]{Times New Roman}
\defaultfontfeatures{Ligatures={TeX},Renderer=Basic} 

% перечень использованных источников
\addbibresource{refs.bib}

% настройка полей
\geometry{top=2cm}
\geometry{bottom=2cm}
\geometry{left=3cm}
\geometry{right=15mm}
\geometry{bindingoffset=0cm}

% настройка ссылок и метаданных документа
% \hypersetup{unicode=true,colorlinks=true,linkcolor=red,citecolor=green,filecolor=magenta,urlcolor=cyan}
\hypersetup{unicode=true,colorlinks=true,linkcolor=black,citecolor=black,filecolor=black,urlcolor=black}

% настройка подсветки кода и окружения для листингов
\usemintedstyle{colorful}
\newenvironment{code}{\captionsetup{type=listing}}{}

% шрифт для листингов с лигатурами
\setmonofont{FiraCode-Regular.otf}[
	SizeFeatures={Size=10},
	Path = templates/,
	Contextuals=Alternate
]

% оформления подписи рисунка
\captionsetup[figure]{labelsep = period}

% подпись таблицы
\DeclareCaptionFormat{hfillstart}{\hfill#1#2#3\par}
\captionsetup[table]{format=hfillstart,labelsep=newline,justification=centering,skip=-10pt,textfont=bf}

% путь к каталогу с рисунками
\graphicspath{{fig/}}

% Внесение titlepage в учёт счётчика страниц
\makeatletter
\renewenvironment{titlepage} {
	\thispagestyle{empty}
}
\makeatother

\counterwithin{figure}{section}
\counterwithin{table}{section}

\titlelabel{\thetitle.\quad}

% для удобного конспектирования математики
\mathtoolsset{showonlyrefs=true}
\theoremstyle{plain}
\newtheorem{theorem}{Теорема}[section]
\newtheorem{proposition}[theorem]{Утверждение}
\theoremstyle{definition}
\newtheorem{corollary}{Следствие}[theorem]
\newtheorem{problem}{Задача}[section]
\theoremstyle{remark}
\newtheorem*{nonum}{Решение}

% настоящее матожидание
\newcommand{\MExpect}{\mathsf{M}}

% объявили оператор!
\DeclareMathOperator{\sgn}{\mathop{sgn}}

% перенос знаков в формулах (по Львовскому)
\newcommand*{\hm}[1]{#1\nobreak\discretionary{} {\hbox{$\mathsurround=0pt #1$}}{}}

% настройки эпиграфов
\setlength{\epigraphwidth}{0.5\textwidth}
\setlength{\epigraphrule}{0pt}

\sloppy

\usepackage{caption}

\makeatletter
\renewcommand{\fnum@figure}{Рисунок~\thefigure} % без точки
\makeatother

\captionsetup[figure]{%
  format=plain,
  justification=centering,
  singlelinecheck=false,
  labelsep=space
}

\addto\captionsrussian{%
  \renewcommand{\contentsname}{\MakeUppercase{Содержание}}%
}

\begin{document}

% Не редактируем: Титульная страница (формируется автоматически из заданной конфигурации)
\begin{titlepage}	% начало титульной страницы

    \begin{center}		% выравнивание по центру
        {Министерство науки и высшего образования Российской Федерации} \\ [0.8cm]

        ФЕДЕРАЛЬНОЕ ГОСУДАРСТВЕННОЕ АВТОНОМНОЕ \\ 
        ОБРАЗОВАТЕЛЬНОЕ УЧРЕЖДЕНИЕ ВЫСШЕГО ОБРАЗОВАНИЯ \\
        «НАЦИОНАЛЬНЫЙ ИССЛЕДОВАТЕЛЬСКИЙ УНИВЕРСИТЕТ ИТМО» \\
        (Университет ИТМО) \\ [0.8cm]
 
	    Институт дизайна и урбанистики \\ [0.5cm]
        \small{Образовательная программа: Генеративные технологии дизайна городской среды} \\

        \small{Направление подготовки: 27.04.07 Наукоемкие технологии и экономика инноваций} \\ [1.2cm]

        \normalsize ОТЧЕТ \\
        \normalsize О НАУЧНО-ИССЛЕДОВАТЕЛЬСКОЙ РАБОТЕ \\ [0.8cm]

        \normalsize по теме: \\
        \normalsize «Формирование морских прибрежных территорий»
    \end{center}
    
    \vfill

    \begin{flushright}

        Студент:\\
        \textit{Выстрчил М.Г., С4110}\\[0.5cm]
        \noindent\underline{\hspace{6.6cm}} \\
        «\underline{\hspace{1cm}}» \underline{\hspace{3cm}} 20\underline{\hspace{1cm}}~г. \\[0.8cm]
        
        Руководитель:\\
        \textit{Воронин Д.В., преподаватель}\\[0.5cm]
        \noindent\underline{\hspace{6.6cm}} \\
        «\underline{\hspace{1cm}}» \underline{\hspace{3cm}} 20\underline{\hspace{1cm}}~г. \\

    \end{flushright}

    
    \vfill % заполнить всё доступное ниже пространство

    \begin{center}
	\normalsize Санкт-Петербург \\
	\normalsize \the\year % вывести дату
    \end{center} % закончить выравнивание по центру

\end{titlepage} % конец титульной страницы


% Не редактируем: Страница содержания (формируется автоматически из section, subsection и пр., указанных в content.tex)
% % Содержание
\begin{center}
  \textbf{СОДЕРЖАНИЕ}
\end{center}
\makeatletter
\renewcommand{\tableofcontents}{%
  \begingroup
    \parindent=0pt
    \parskip=0pt
    % НИКАКОГО \section*{\contentsname}
    \@starttoc{toc}%
  \endgroup
}
% \makeatother

% \pagestyle{empty}

% \clearpage
\setcounter{page}{1}   % страница с оглавлением = 1
\pagestyle{plain}      % включить вывод номера
\tableofcontents
\clearpage

% \tableofcontents
\newpage


% Редактируем: всё остальное: вступление, др. этапы, заключение, приложение

\section*{СПИСОК СОКРАЩЕНИЙ И УСЛОВНЫХ ОБОЗНАЧЕНИЙ}
\addcontentsline{toc}{section}{СПИСОК СОКРАЩЕНИЙ И УСЛОВНЫХ ОБОЗНАЧЕНИЙ}

В представляемой работе были использованы следующие сокращения и обозначения:

\begin{itemize}

    \item \textit{ВК РФ} - Водный кодекс Российской Федерации;
    
    \item \textit{ГИС} - географическая информационная система;
    
    \item \textit{ГПЗУ} - градостроительный план земельного участка;

    \item \textit{ГрК РФ} - Градостроительный кодекс Российской Федерации;
    
    \item \textit{ЕГФД} - Единый государственный фонд данных о состоянии окружающей среды;
    
    \item \textit{ЗК РФ} - Земельный кодекс Российской Федерации;

    \item \textit{КоАП РФ} - Кодекс Российской Федерации об административных правонарушениях;

    \item \textit{ПЗЗ} - Правила землепользования и застройки;

    \item \textit{РФ} - Российская Федерация;
    
    \item \textit{CVI} - Coastal Vulnerability Index (Индекс уязвимости береговой зоны);

    \item \textit{DEM} - Digital Elevation Model (Цифровая модель рельефа);
    
    \item \textit{DSAS} - Digital Shoreline Analysis System (Система цифрового анализа береговой линии);
    
    \item \textit{IPCC} - Intergovernmental Panel on Climate Change (Межправительственная группа экспертов по изменению климата (МГЭИК));
    
    \item \textit{OSM} - OpenStreetMap;

    \item \textit{PCA} - Principal Component Analysis (Метод главных компонент);
    
    \item \textit{USGS} - United States Geological Survey (Геологическая служба США).

\end{itemize}

\newpage
\section*{ТЕРМИНЫ И ОПРЕДЕЛЕНИЯ}
\addcontentsline{toc}{section}{ТЕРМИНЫ И ОПРЕДЕЛЕНИЯ}

\begin{itemize}
\item \textit{Абразия} (от лат. abrasio - «соскабливание, сбривание») - это геоморфологический процесс механического разрушения берегов водоёмов (морей, океанов, озёр, водохранилищ) волнами, течениями и прибоем.

\item \textit{Берег} - полоса суши, на которой имеются формы рельефа и накопления наносов, созданные волнением при современном среднемноголетнем уровне воды \cite{sp_277_1325800_2016}.

\item \textit{Береговая зона} - зона, состоящая из трех геоморфологических элементов: берега, подводного склона и пляжа  \cite{sp_277_1325800_2016}.

\item \textit{Береговая линия} - среднемноголетнее положение уреза воды \cite{sp_277_1325800_2016}.

\item \textit{Берегозащитное (берегоукрепительное) сооружение} - гидротехническое сооружение для защиты берега от размыва и разрушения \cite{gost_r_59241_2020, sp_277_1325800_2016}.

\item \textit{Буна} - пляжеудерживающее сооружение для удержания наносов из естественного вдольберегового потока наносов и сохранения естественного или искусственного пляжа в межбунных отсеках \cite{sp_32_103_97}.

\item \textit{Вдольбереговое перемещение наносов} - явление массового однонаправленного перемещения наносов вдоль берега, называемое также их продольным перемещением, которое происходит при подходе волн под острым углом к берегу и обуславливается наличием вдольбереговой составляющей потока волновой энергии или под воздействием течений неволновой природы \cite{sp_32_103_97}.

\item \textit{Вдольбереговой поток наносов} - однонаправленное результирующее перемещение наносов вдоль берега за длительный интервал времени \cite{sp_277_1325800_2016}.

\item \textit{Ветровые течения} - течения на водной поверхности, создаваемые касательными напряжениями, вызванными действием ветра \cite{sp_277_1325800_2016}.

\item \textit{Динамика береговой зоны} - совокупность береговых процессов по перестройке берега и подводного берегового склона \cite{sp_277_1325800_2016}.

\item \textit{Искусственный пляж} - пляж, созданный при участии антропогенных \\ средств доставки наносов в береговую зону (относится к гидротехническим сооружениям, может использоваться как в берегозащитных, так и рекреационных целях) \cite{sp_277_1325800_2016}.

\item \textit{Пляж} - форма рельефа береговой зоны (природная или искусственно созданная), сложенная наносами, образованная в зоне действия прибойного потока \cite{gost_r_59241_2020}.

\item \textit{Пляжный материал} - совокупность несцементированных обломочных осадков различной зернистости (песок, гравий, галька), аккумулирующихся в зоне берегового вала и подводного профиля пляжа под воздействием волновых, течений и литодинамических процессов.

\item \textit{Пляжная подсыпка} (англ. beach nourishment, beach replenishment, beach fill) - это искусственное размещение пляжного материала в прибрежной зоне для восполнения потерь наносов, вызванных естественной эрозией и абразией.

\item \textit{Размыв} - перемещение грунтов морского дна, пляжа, элементов наброски или крепления откоса от воздействия волн и течений \cite{gost_r_59241_2020}.

\end{itemize}

\newpage
% \section*{ВВЕДЕНИЕ}
\begin{center}
  \textbf{ВВЕДЕНИЕ}
\end{center}
\addcontentsline{toc}{section}{ВВЕДЕНИЕ}

Морская береговая линия, являясь местом совокупного взаимодействия ансамбля гео-, гидро- и аэромеханических процессов, представляет собой сложную и динамическую систему, обладающую, в результате влияния внешних сил, высокой степенью потенциального риска деформаций прибрежных территорий.
Указанные деформации, в свою очередь, влекут за собой утрату конструкционной целостности находящихся в прибрежных зонах инженерных и природных объектов, что, помимо экономического ущерба, провоцирует опасность и для человека.

Эффективное управление этими процессами требует применения различных средств гидротехнических средств защиты, рациональный выбор которых определяется в ходе объемных, специфичных и высококвалифицированных работ по гидротехническому проектированию.
Примененные к проектированию прилагательные, в совокупности с ограниченными количеством квалифицированных инженеров-гидротехников, не позволяют распространить в полной мере опыт этих работ на всем протяжении береговой линии, в связи с чем они локализуются в промышленных прибрежных зонах.

При этом естественная тяга человека к водным объектам обуславливает заполнение прибрежных территорий наравне с объектами промышленного и инфраструктурного назначения рекреационными и жилыми зонами.
Помимо гуманитарной направленности, перечисленные объекты объединяет их относительно большое количество, что влечет за собой их проектирование организациями, имеющими ограниченный гидротехнический опыт.
Последствия ограниченной компетентности разработчиков и исполнителей работ в прибрежных зонах проявляются в преждевременной утрате эксплуатационных характеристик сооружаемыми объектами.
В то же время, распределение проявления негативных явлений во времени, в совокупности с неравномерностью распределения зон повышенного риска вдоль береговой линии, создают у внешнего наблюдателя иллюзию естественности и неизбежности этих процессов, выражающуюся в молчаливом принятии тезиса "вода камень точит".
Лингвистическим антагонистом этого утверждения является известное "под лежачий камень вода не течет", что формирует дугу диалектического противоречия, внести свой вклад в разрешение которого преследует, в идейном плане, цель данной работы.

Рассматривая практическую плоскость вопроса, следует опираться на тот факт, что основная масса удобных для своего освоения участков береговой линии была занята к настоящему моменту в ходе эволюционного заполнения прибрежных земель городским пространством.
Оставшиеся свободными для развития прибрежные территории требуют выполнения мероприятий по оценке своего потенциала.
Для обеспечения экономической эффективности освоения прибрежных территорий одним из факторов для такой оценки должен являться риск проявления негативных явлений, связанных с гидромеханическими процессами, накладывающими на землепользователя дополнительные траты на обустройство берегозащитных конструкций или ликвидацию последствий в случае пренебрежения этими мероприятиями.
В любом случае непреклонные силы природы возьмут свое, вопрос лишь в том, какую степень риска готовы принять ответственные землепользователь и государство, использующие свои ресурсы к достижению коллективной пользы.

Опираясь на сказанное, разработка общедоступного метода оценки прибрежных морских территорий по степени риска проявления негативных деформационных и гидромеханических процессов является актуальной задачей, решение которой повысит как экономическую эффективность мероприятий по их освоению, так и безопасность и сохранение названных земель при использовании.

\vspace{1em}
\textbf{Формулировка проблемы}

В градостроительных отношениях применяются методы формирования морских прибрежных территорий, которые не учитывают гидромеханических и геомеханических процессов вдоль морской береговой линии, что впоследствии может привести к созданию потенциально опасной ситуации для жителей города и повлечь за собой негативные экономические эффекты.


% Отсутствие доступной информации о локализации зон повышенных рисков проявления негативных гидромеханических и геомеханических процессов вдоль морской береговой линии провоцирует ошибки при проектировании и эксплуатации объектов на прибрежных территориях, что влечет за собой негативные экономические последствия для землепользователей и потенциальную опасность для жителей города.

\vspace{1em}
\textbf{Объект исследования}

Система морских прибрежных территорий, планируемая к градостроительному освоению.

% Морские прибрежные территории.

\vspace{1em}
\textbf{Предмет исследования}

Метод оценки морских прибрежных территорий по степени риска проявлений негативных гидромеханических и геомеханических процессов.

\vspace{1em}
\textbf{Цель исследования}

Разработка и внедрение метода оценки морских прибрежных территорий по степени риска проявлений негативных гидромеханических и геомеханических процессов.

% Оценка морских прибрежных территорий по степени риска проявлений негативных гидромеханических и геомеханических процессов.

\vspace{1em}
\textbf{Идея работы}

Оценка риска негативных гидромеханических и геомеханических процессов за счет экстраполяции элементов гидротехнического опыта анализа нагрузок на береговую линию и уточнение за счет вычисляемого индекса уязвимости береговой зоны (CVI).

\vspace{1em}
\textbf{Задачи исследования}

\begin{enumerate}
    \item Анализ предметной области, научный поиск и структурирование материала по предмету исследований.
    \item Определение метрики (индекса) для описания уязвимости морской прибрежной территории.
    \item Проведение анализа состава и доступности необходимых исходных данных.
    \item Анализ методологии сбора и обработки геопространственных данных для вычисления этого индекса.
    
    % \item Установление критерия, позволяющего классифицировать прибрежные территории по степени риска;
    % \item Вычисление предложенного индекса для морских прибрежных территорий РФ;
    % \item Апробация полученных результатов;
    % \item Разработка рекомендаций по формированию морской прибрежной территории с учетом риска проявления негативных гидромеханических и геомеханических событий.
\end{enumerate}

\vspace{1em}
\textbf{Методология и методы}

\begin{enumerate}
    \item Методы научного поиска и анализа.
    \item Методы сбора и обработки геопространственных данных.
    \item Методы математической статистики.
    \item Методы геопространственного анализа.
\end{enumerate}

\vspace{1em}
\textbf{Новизна исследования}

Разработанные в ходе исследования методики получения геопространственных данных и вычисленные для морских прибрежных территорий индексы уязвимости позволят повысить качество научного описания объекта исследований, формируя при этом новые знания о мире.

\vspace{1em}
\textbf{Теоретическая и практическая значимость}

\begin{enumerate}
\item Разработанные в ходе исследования методики получения геопространственных данных смогут быть использованы в смежных по тематике научных и инженерных работах;

\item Сформулированные рекомендации к процессу территориального планирования позволят учитывать специфику морских прибрежных территорий, повышая тем самым экономическую эффективность принимаемых решений. 
\end{enumerate}

\vspace{1em}
\textbf{Соответствие направлению подготовки}

Выполненный в ходе НИР анализ уязвимости морских береговых линий повысит эффективность территориального планирования, позволяя оптимизировать затраты на гидротехнические сооружения, время на подготовку проектных материалов, трудовые ресурсы, необходимые для их формирования, и сократить, в перспективе, траты на ликвидацию негативных гидромеханических и геомеханических событий, что согласуется с паспортом направления 27.04.07 «Наукоемкие технологии и экономика инноваций».



\newpage
\section{Основные понятия, сущности и определения}

\epigraph{
    Если выпало в~Империи родиться,\\
    лучше жить в~глухой провинции у~моря.
}{\textit{«Письма римскому другу»}\\И.~Бродский}


Невозможно строго доказать, но, наверное, каждому будет легко внутри себя согласиться с тем, что стремление человека к воде является естественным.
С физической точки зрения вода -- это второй после воздуха элемент, без которого невозможно существование высокоразвитых живых организмов.
С метафизической точки зрения водные объекты -- неиссякаемый источник внутреннего вдохновения и отдохновения, наполняющий человека спокойствием и гармонией как по отношению к себе самому, так и окружающим.
С экономической, именно близость к водным объектам позволила людям раскрыть свой потенциал, на всем протяжении истории открывая перед смотрящим новые горизонты своего использования.

Уникальность береговой линии, как  главного элемента, формирующего вокруг себя прибрежные территории, состоит в том, что она является местом пересечения основных природных сил: воды, земли и воздуха. Добавив к ним (для поэтической полноты) силы огня, проявляющиеся в форме теплоты солнечного света, мы получим практически сакральное место в быту, носящее скромное имя: \enquote{пляж}. 
Переведя это на сухой язык науки, объект исследований можно определить как место, формируемое одновременным проявлением гидродинамических, геомеханических, аэрологических и термодинамических процессов, определяющих необходимость междисциплинарных знаний для своего изучения и управления.

Необозримый промежуток времени проведенного людьми в таком соседстве не мог не сформировать внутренние противоречия и вызовы, таящие в себе наравне с очевидными благами специфичные опасности, реагировать на которые было вынуждено человеческое общество. Глубина необходимой, как, впрочем, и доступной, для человека рефлексии происходящих вокруг него природных и антропогенных процессов до сих пор не позволяет в полной мере разрешить полный корпус противоречий, формирующихся внутри такого важного, интересного и ценного ресурса как прибрежные территории.


В ограниченных рамках представляемой работы невозможно в полной мере отразить богатую историю взаимодействия человека с прибрежными территориями, отчего дальнейшее конспективное изложение преследует своей целью лишь сфокусировать интерес читателя к важности изучения данного вопроса. 
Сложность явлений происходящих вокруг причалов, набережных и пляжей определяет актуальность и необходимость данной работы в контексте гармоничного развития урбанизированных прибрежных территорий.  



% \subsection{Исторический контекст формирования системы взаимодействия человека с прибрежными территориями}

\subsection{Исторический контекст формирования системы \mbox{взаимодействия человека с прибрежными территориями}}


\subsubsection{Глубокая древность}

Расселение человека вдоль водных объектов с начала существования его как биологического вида было жизненной необходимостью.
Отсутствие средств аккумуляции, переноса и хранения воды грозило человеку, как впрочем и любому животному, очевидными угрозами для здоровья и жизни в случае его отрыва от источника пресной воды.
Необходимость присутствия водных объектов в пешей доступности превратила береговые линии в естественный ориентир, вдоль которого первые люди провели свое расселение.
Экстенсивный характер природопользования первых людей накладывал на них необходимость обеспечения каждого из субъектов большой территорией, что в свою очередь послужило стимулом к расселению человечества по всем пригодным для естественного проживания территориям Земли.  

Тысячелетия эвристического поиска позволили человечеству перейти от присваивающей к производящей модели экономики совершив при этом первую в истории революцию -- неолитическую.
Формирование сельского хозяйства позволило перейти к оседлому образу жизни, сформировав не только базис для роста популяции, но и прибавочный продукт, позволивший в перспективе обеспечить разделение труда.
Развитие этих процессов определило удобство группового проживания людей на локальных участках местности сформировав первые города, определившие своим названием само явление \enquote{цивилизации}.

Однако примитивность древних технологий позволяла лишь использовать ограниченные природные блага, локализуясь в местах их естественного происхождения.
Первые цивилизации: Месопотамии в междуречье Тигра и Ефрата, египетские царства Нила, древние китайские цивилизации в долине рек Янцзы и Хуанхэ, Хараппская цивилизация в долине Инда, цивилизации Амазонии и Мезоамерики -- все они стали возможны благодаря естественным иловым удобрениям земель вследствие периодичных разливов прилегающих к ним рек.
Блага такого соседства были для людей очевидны, однако в противовес этому древние люди подвергали себя риску экстремальных разливов рек, дошедших до нас в общераспространенном и архетипическом мифе о \enquote{великом потопе}.
Этот генетический страх беззащитности перед водной стихией до сих пор живет в глубине нас, определяя эмоциональную вовлеченность, начиная от современных религиозных текстов и заканчивая детскими сказками о Мумми-троллях.
Все эти сюжеты, определяя надежду на возрождение после мировой катастрофы, стимулировали людей искать собственные способы защиты от природных сил.

Осознав свою силу над природой, человечество начало трансформировать окружающий себя водный каркас, производя искусственную мелиорацию прилегающих земель и повышая тем самым урожайность своих земель.
Произведенные избыточные продукты стали условием обмена и бартерной торговли между соседствующими культурами и народами.
С этого момента и до наших дней водный транспорт является наиболее экономически эффективным в плане доставки грузов.
Начав от сплавов по рекам, каботажные плавания вдоль берегов сшили разрозненные  города и полисы Средиземноморья между собой, став одним из условий существования Великой Римской Империи.

Не являясь в этом первыми, именно римляне достигли недосягаемых для своего времени высот в вопросах управления водными ресурсами.
Лишь вскользь упомянув выходящий за рамки данной работы факт сооружения акведуков, сконцентрируемся на примерах морских гидротехнических сооружений.
Известный опыт сооружения морской дамбы в период Третьей Пунической войны позволил в итоге римлянам замкнуть кольцо осады Карфагена, отчасти повторив опыт Александра Македонского при осаде Тира.
Освоив технологии бетонных работ, римский гений применил его при сооружении пирсов в порту Косса, волнозащитных моллов в Остии и Цезарии, однако, несмотря на монументальность выполненных работ, природа побеждала и разрушала инженерные объекты \cite{mccann1987roman}.

Климатические изменения, наравне с кризисом рабовладельческой экономической модели экономики, погрузил человечество с темные средние века.
Несмотря на то, что необходимость гидротехнических работ не была забыта при феодальной формации качественный прогресс в этом вопросе был достигнут лишь при наступлении Нового времени. 

\subsubsection{Старая Голландия и Новое время} 

Одним из наиболее захватывающих и вдохновляющих примеров систематической работы над трансформацией морских прибрежных территорий является Нидерландский опыт, длившийся на протяжении двух тысячелетий и не останавливающийся в наши дни.
Подробный очерк, описывающий основные вехи этого противостояния человека и моря, описан в прекрасной работе \cite{Alahverdi2016}, ключевые фрагменты которой будут определять дальнейший текст.

Антропогенное влияние человека, выражавшееся в вырубке лесов под личные нужды и высвобождении земель под пахотные угодья, стало катализатором эрозионных разрушительных процессов, усугублявшихся нагонными волнами из Северного моря и общим повышением уровня Мирового океана.
Море стало неприклонно забирать землю.
Начиная с VI века до нашей эры прибрежные жители стали формировать искувсственные возвышения для защиты своих жилищ, развивая свой успех последующим сооружением дамб и насыпей.
Опуская подробности этой постоянной, каждодневной борьбы, состоящей из череды сменяющейся побед и поражений человека перед природой, можно выделить переломную точку в катастрофическом наводнении 1421 г.
Считается, что именно оно послужило стимулом к переходу от разрозненных персональных усилий к централизации мероприятий по терраформированию прибрежных территорий \cite{Alahverdi2016}.

Лишь к XVI веку, благодаря разработке инновационных ветряных мельниц и адаптации их для откачки воды, удалось приступить к последовательному осушению польдеров.
Человек перешел от защиты к нападению и в результате победил (рисунок \ref{pic:Windmill}).

\begin{figure}
    \begin{center}
	\includegraphics[width=150mm]{fig/The_Windmill_at_Wijk_1670_Ruisdael.jpg}
	\caption{-- «Мельница около Дюрстеде» картина Якоба ван Рейсдала, 1670~г.}
	\label{pic:Windmill}
    \end{center}
\end{figure}

На протяжении следующих веков сооружаемые дамбы служили как инструментом активной обороны в многочисленных европейских войнах, так и источником опасностей для жителей Северной Голландии, не исчерпавшим себя до конца вплоть до наших дней.
Известные своими ужасающими последствиями наводнения Зейдерзе (1916 г.) и прорывы дамб в дельте Рейна и Мааса (1953 г.) стали катализатором развития всемирно известной голландской геомеханической и гидротехнической школы, опыт который позволил предсказать и избежать многочисленные техногенные и природные катастровы по всему миру.

Современный Нидерландский опыт будет не раз упоминаться далее при анализе научных источников по рассматриваемой теме.
Искупить недостойно скудное и поверхностное перечисление Нидерландского опыта гидротехнических работ, в текущем разделе, можно дополнительным акцентированием внимания на то, что именно им принадлежит заслуга перехода от сугубо эвристического опыта взаимодействия с прибрежными территориями к сугубо научному, и пока еще естественному для нас, восприятию окружающего мира.
Сформированная в период Нового времени философия строго научного познания сформировала весь окружающий нас мир.
Символично, что именно из Голландии силой, энергией и волей Петра I свет этого знания перешел на территорию России, об опыте которой речь пойдет далее.

\subsubsection{Гидротехнический опыт Российской империи}  

Важнейшей вехой в истории освоения водных объектов стало правление Петра I, неразрывно связавшее будущее России с флотом, морем и западным миром.
К моменту его воцарения Россия не имела выхода к внешним морям, кроме выхода к Белому морю у г. Архангельска.
Удаленный замерзающий на большую часть года Архангельский порт уже к тому моменту не мог полностью удовлетворить потребности России во внешней торговле \cite{GapeevKononov2009}, что потребовало проведения срочных реформ.

Опыт и специалисты привезенные Петром I из Голландии на русскую землю позволили в считанные годы сократить экономическое отставание от соседних государств, создать первый русский флот и победить Швецию в Северной войне.
Период правления Петра I -- это период форсированного развития, период проб и ошибок -- начавшийся с ограниченного успеха и утраты Азова и Таганрогской гавани и увенчавшийся успехом градостроительного опыта в дельте Невы.
Кронштадтские укрепления на острове Котлин, галерная гавань на Васильевском острове и торговый порт на его стрелке стали примером первых берегоукрепительных и дноуглубительных работ, создания новых насыпных территорий на территории известного нам Санкт-Петербурга \cite{ChuneevaIzbasarova2021}.
Большие инфраструктурные проекты примерами которых являются Вышневолоцкая, Мариинская и Ивановская канальные системы были начаты при нем и закончены его преемниками \cite{GapeevKononov2009}.

Недолгое, но яркое правление Петра I прошло под девизом скорости и рациональности в ущерб классической монументальности ассоциирующейся с периодом начавшимся во второй половине XVIII века.
Эта \enquote{эпоха гранита}, оставленного отступившим оледенением дошла до нас в виде набережных сковавших берега Невы и городских каналов, защитивших постройки от наводнений и оползневых процессов, повысив при этом допустимую плотность застройки города.

Закрепившись на Балтийском море военные, а вслед за ними и гражданские силы, были направлены на юг к прибрежным территориям Черного и Азовского морей.
Плоды побед в Русско-Турецких войнах расцвели в формах русских портов Одессы и Новороссийска.

Эпоха пара, бетона и механизации производств, стартовавшая в России со второй половины XIX века, превратила прибрежные территории в индустриальные, связанные между собой не только водным и канальным транспортом, но и железными дорогами.
Созданные производства не могли существовать без воды и людей, определив тем самым как свое непосредственное расположение, так и урбанизированный ландшафт доходных домов, окружающих исторический центр городов \cite{BelyaevaBelyaev2021}.

В целях совершенствования качества прибрежных территорий также производились масштабные дноуглубительные работы (рисунок \ref{pic:МорскойКанал1885}), потребовавшие для своего проведения, помимо использования производительной техники, подготовки узкоспециализированных специалистов гидротехников и геодезистов, повысив тем самым доступный жителям уровень технического образования.

\begin{figure}
    \begin{center}
	\includegraphics[width=150mm]{fig/МорскойКанал1885}
	\caption{-- План Морского канала и порта на Гутуевском острове, 1885~г.~\cite{ChuneevaIzbasarova2021}}
	\label{pic:МорскойКанал1885} % название для ссылок внутри кода
    \end{center}
\end{figure}


Развивающиеся железные дороги укрепили связь между западной Россией и Дальним Востоком. 
В 1897 г. началась закладка первых бетонных массивов для устройства причалов портовых сооружений в бухте Золотой Рог Владивостока, велись дноуглубительные работы вдоль фарватера Амура.
В 1898 г. Россия получает в аренду Квантунский полуостров и начинает работы по созданию Порт-Артурской крепости и коммерческого порта Дальний, утраченных в скором времени в результате поражения в Русско-Японской войне.

Неким «Арктическим хвостом» этого периода истории стало обустройство в разгар Первой мировой войны незамерзающего порта в Мурманске, как реакции на риски морской блокады со стороны Балтийского моря.

Подводя итог этому историческому периоду, нельзя упрекнуть государственных и промышленных деятелей в пренебрежительном и легкомысленном отношении к прибрежным территориям.
Несмотря на гигантскую площадь Российской империи доступные к использованию морские прибрежные территории оставались весьма ограниченными: замыкаясь внутренними морями Новороссии, Балтийским морем, полуокруженным скандинавскими странами, суровыми северными портами Архангельска и Мурманска, молодыми городами Дальнего Востока.
Вызовы освоения северного морского пути перешли по наследству Советской власти, благодарно принявшей их наравне с результатами обширного накопленного гидротехнического опыта, ценимого как внутри страны, так и и за ее пределами \cite{Peterson2016}.

\subsubsection{Советский период}

Советский период отечественной истории неизбежно ассоциируется с прогрессивными инновациями во всех областях жизни общества.
Не явилась исключением и область освоения прибрежных территорий морей и рек нашей страны.

Общественный характер производства и использования ресурсов позволил Советскому государству перейти от стихийного и сугубо утилитарного использования прибрежных территорий к планомерному, последовательному освоению.
Фактически это позволило советской архитектурной школе рассматривать прибрежные гидротехнические объекты как часть архитектурного ансамбля и фасада прибрежных городов.
В работе \cite{Rubleva2022} подробно описана история развития концепции «Морского фасада» Ленинграда, предполагающую намыв территорий на Васильевском острове, спрямление берегов и строительство гранитных набережных, формирующих новое «лицо» города, обращенное к Балтике.
Элементы этого опыта были использованы советскими архитекторвами участвовавшими в разработке проектов восстановления разрушенных в период Великой Отечественной Войны черноморских городов \cite{VasilievOvsyannikova2019}.

Непрекращающийся рост советской экономики определил необходимость расширения портовых мощностей, реализовавших себя в создании новых глубоководных терминалов, примерами которых могут служить Находка и Ильичевск \cite{Sabaydash2023}, разгружающие порты Владивостока и Одессы соответственно.
Воля и смелость советского человека основала города и поселки Арктики: Диксон, Тикси, Певек, обеспечивая возможность полярной навигации и контроль наших северных границ.

Уникальность перечисленных мероприятий не позволила бы им состояться без привлечения мощной гидротехнической школы сконцентрированной внутри созданных отраслевых институтов: Союзморниипроекта, Ленморниипроекта, Новоморниипроекта и пр.

С 1960-х годов фокус внимания советской гидротехники распространяется на комплексное использование морских берегов, обеспечивая их рекреационную функцию.
Советские инженеры (В.П. Зенкович и др.) разработали теорию «свободных пляжей» как наилучшей защиты берега, что позволило сохранить курортную функцию городов при интенсивной эрозии береговой линии.
Внедрение унифицированных железобетонных элементов для берегоукрепления позволило обеспечить масштабное строительство бун, волноломов и искусственных пляжей в Сочи, Крыму и Прибалтике \cite{Kuklev2003}.

Однако высокие темпы развития прибрежных морских территорий не обошли и досадные ошибки, допущенные при стратегическом планировании этих территорий.
Разобщенность между ведомствами которым подчинялись порты (Минморфлоту) и горисполкому, определявшему городское устройство, приводило к дисбалансу в принимаемых решениях и конфликтах \enquote{город-порт}.
Пример Мурманска наглядно демонстрирует, к чему приводит победа в подобных спорах порта, фактически отрезавшего доступ для жителей к береговой линии.

Подводя итог очередному рассматриваемому периоду отечественной истории, можно с одновременной гордостью и грустью признать его \enquote{золотым веком} в области науки, культуры и практики в отношениях между человеком и природой морских прибрежных территорий.
К сожалению, богатое наследство гидротехнических сооружений, оставленных нам предками, к настоящему времени ведет одинокую и обреченную борьбу со стихией без должной поддержки со стороны властей.
Рассмотрению подробностей развития этой борьбы посвящен заключительный этап исторического обзора.

\subsubsection{Наше время}

Распад Советского Союза в 1991 году не только радикально изменил внутреннее устройство входящих в него республик, но и внешние, в том числе и морские, границы ставших независимыми стран.
Потеря значительной части портовой инфраструктуры на Балтике и Черном море, преобразование внутреннего Азовского и условно внутреннего Каспийского моря в международные акватории трансформировали условия для освоения прибрежных территорий. 
За три с лишним десятилетия постсоветского периода Россия прошла путь от острого кризиса в управлении прибрежным комплексом до восстановления и формирования новой стратегии освоения морских пространств.

Одной из краеугольных причин, формирующих проблемы в сфере использования прибрежных территорий, стало усложнение модели взаимодействия между выгодопользователями в рамках рыночных отношений, усугубившее проблему взаимодействий \enquote{город-порт}, добавив в нее еще и \enquote{собственника}.
Превратив прибрежные территории в условиях рыночной экономики в ценный земельный ресурс для жилого строительства \cite{ShcheglovaDokhov2025}.
Печально, что одним из главных пострадавших в этих спорах стал безмолвный городской обыватель, фактически вывеленный из участия в принятии решений по использованию прибрежных территорий.

Вдоль береговых линий было запущено \enquote{право сильного} и богатого.
В результате возник парадокс -- с формальной точки зрения Водный кодекс РФ \cite{WaterCodeRF} гарантирует свободный доступ к воде, запрещая приватизацию участков и строительство в границах 20 метров от берега, очевидно допуская все это за этим буфером.
С другой стороны, согласно КоАП РФ ст. 8.12.1 штрафы за нарушение режима водоохранной зоны и закрытие доступа граждан к береговой полосе предусматривает штраф 3000–5000 руб. для гражданских лиц.
Сложно представить, но до сих пор, несмотря на суровость наказания, вдоль берегов нашей родины существуют отдельные лица, идущие на риск подобных взысканий, де факто присваивая себе общую землю.

Расщепление внимания власти и попустительство при привлечении землепользователей к ответственности привело к абразии, оползням и размывам берегов, что привело к сокращению ширины пляжей и отступанию берегового уступа \cite{Kuklev2003}.
Наглядный пример, отражающий взаимодействие отдыхающих с остатками пляжей Черного моря, представлен на рисунке \ref{pic:Пляж_в_Лазоревском}.

\begin{figure}
    \begin{center}
	\includegraphics[width=150mm]{Пляж_в_Лазоревском.png}
	\caption{-- Берег на северной окраине пос. Лазаревское (г.~Сочи)~\cite{SochiCoastal2018}}
	\label{pic:Пляж_в_Лазоревском}
    \end{center}
\end{figure}

Прогрессивные действия по наведению порядка и культуры использования прибрежных территорий начались только в XXI веке.
Освоение нефтегазовых шельфов в рамках проектов Сахалин-1 и Сахалин-2, развитие Сан\-кт-Петер\-бург\-ского большого порта, инфраструктурные проекты Владивостока на острове Русский были оттенены масштабом подготовки к Сочинской зимней олимпиаде 2014 года.  
Олимпийский проект сделал Сочи самым дорогим в истории и радикально усилил туристско-рекреационный потенциал региона \cite{Mishulina2014}.

После воссоединения Крыма с Россией в 2014 году главным инфраструктурным проектом стал \enquote{Крымский мост}, строительство которого было завершено в мае 2018 года.
В то же время был запущен, длящийся до сих пор, процесс перераспределения туристического потока с заграничных курортов \cite{Ivanenko2024}.
По состоянию на 2020 год Причерноморье имеет самую высокую в России плотность туристической инфраструктуры.
В ходе оценки туристического потенциала Краснодарского края была выполнена масштабная классификация пляжей: более 70 пляжных зон официально проклассифицированы в Причерноморье, только в Сочи функционирует свыше 130 пляжей.
Азовское побережье (применительно к границам РФ 2021 года) протяженностью 550 км ежегодно принимало более 2,5 млн туристов \cite{Volkova2021}. 
При этом порядка 230 км берега вдоль Тамани находятся под угрозой абразии и оползней \cite{Misirov2024}.

Согласно исследованиям \cite{Kosyan2017} система комплексного управления береговой зоной российского сектора Чёрного моря фрагментарна: разные ведомства отвечают за землепользование, экологию, гидротехнику; отсутствует единый координационный центр. Недостаточно развита нормативная база «береговой полосы безопасности» и режима ограничения застройки, слабо внедрён кадастр морских берегов и современные системы мониторинга \cite{Gogoberidze2024}.

Подводя итог нельзя не признать, что в настоящее время ведется активная научная работа в области определения эффективных и честных компромиссов между государством и интересантами земельных отношений морского побережья, наведение правового порядка в этой сфере.
Однако объем накопленных проблем не позволяет решить их единовременно.
Принимая во внимание обстоятельства необходимости включения в плановую работу над формированием и восстановлением береговой линии воссоединяемых с Россией территорий, переоценить важность научного изучения вопросов взаимодействия города, человека и прибрежных территорий между собой \cite{Kruglova2025}.

\subsection{Современный мировой опыт формирования и анализа \mbox{прибрежных пространств}}


Очевидно, что было бы искажением действительности ограничивать распределение проблем взаимодействия человека, города и моря только отечественной географией, так как в разной степени они свойственны для всех стран, имеющих выход к морскому и океаническому берегу.

\begin{figure}
    \begin{center}
	\includegraphics[width=\textwidth]{fig/DeltaWorks.jpeg}
	\caption{-- Проект Delta Works (Нидерланды)~\cite{deltaprogramme_southwest_delta}}
	\label{pic:DeltaWorks}
    \end{center}
\end{figure}

Для определенных территорий проблемы вырождаются в риск наводнений и цунами, пригоняемых штормами.
Так система дамб "Delta Works" (рисунок \ref{pic:DeltaWorks}) в Нидерландах является одним из крупнейших комплексов берегозащитных проектов мира \cite{Pilarczyk2012}, защищающим Роттердам и дельту Рейна–Мааса от штормовых нагонов.
Посоревноваться, в перспективе, с ней сможет проектируемая система "Texas Coastal Barrier" для защиты побережья Техаса и Хьюстона от ураганов \cite{Rasmussen2023}.
Классические берегозащитные укрепления и наращивание площадей пляжей реализуются в рамках защиты Бактонского газового терминала в Норфолке и прибрежных поселков в Английском Ланкашире (проекты "Great Sea Wall" и "Rossall \& Anchorsholme Coastal Defence Scheme").
Однако самым масштабным проектом является \enquote{Великая Японская стена} протяженностью более 400 км, строительство которой было интенсифицировано после разрушительного цунами 2011 года \cite{Wachter2023}.

Масштабные проекты по \enquote{отвоевыванию} земли у моря реализуются в Лагосе (Нигерия) "Eko Atlantic City" \cite{Wasiu2021}, индонезийской Джакарте "The Great Garuda" \cite{EkaPermanasari2019}, в программах "Land Reclamation", предполагающей крупные отсыпные проекты Гонконга и Шанхая (Chek Lap Kok, Lantau, Pudong и т.д.)\cite{Lai2019}.
Применим здесь и обширный Санкт-Петербургский опыт формирования намывных территорий.

Научный интерес многих исследователей также сконцентрирован относительно более локальных проектов.
Так в работе \cite{EscuderoCastillo2018} разбирается, как застройка на барьерном острове курорта Канкун (Мексика) привела к эрозии пляжа и утрате его защитных экосистемных свойств.
Авторы \cite{Oliveira2024BeachNourishment} методами математического моделирования исследовали влияние шторма Hercules 2014 года на эффективность береговой защиты пляжа Кошта-да-Капарика (около Лиссабона) и обосновали объемы восполнения песчаного пляжного материала.
Вторит этой идее и Джеймс Хьюстон (один из ключевых американских специалистов по берегозащите): в своей работе \cite{Houston2022BeachNourishment} он получил схожие результаты при анализе воздействия урагана Sandy.
Рассмотрев три стратегии защиты побережья: удаление застройки вглубь суши, строительство инженерных сооружений (буны, моллы, волноломы и т.п.) и мероприятия по пляжной подсыпке автор обосновал, что именно последний вариант даёт наилучшее сочетание снижения ущерба, устойчивости и окупаемости затрат на свое производство.
Оценка автора показывает, что проекты подсыпки под управлением Корпуса инженеров США позволили избежать порядка 1,3 млрд долларов ущерба прилегающей к морю инфраструктуре \cite{Houston2022BeachNourishment}.

Основываясь на данных дистанционного зондирования Земли, исследование прибрежной зоны пляжа Келананг \cite{MatIsa2023CoastalKelanang} определило связь эррозии и сокращения пляжной зоны с вырубкой мангровых деревьев вдоль берега. 
Предлагая использовать комплекс гидротехнических и \enquote{мягких} мер, включающих реозеленение пляжной зоны, авторы подчеркивают важность междисциплинарного подхода (ландшафтные архитекторы, инженеры, экологи) в решении аналогичных проблем возникающих в мире.

Ранжирование берегозащитных мер, наилучшим образом подходящих для конкретных типов берегов и условий среды выполнено в работе \cite{Sauve2022CoastalDefence}.
Обобщив опыт 411 научных исследований авторы предлагают методику \enquote{динамичного анализа условий} для выбора наиболее эффективных мер защиты.
Схожие идеи транслируются через труд \cite{Huynh2024HybridCoastalDefence}, в котором на основе анализа 304 исследований (875 наблюдений) были выделены четыре типа мер защиты:

\begin{itemize}
\item \textit{hard} -- \enquote{жесткие} меры (дамбы, волноломы и т.п.);
\item \textit{soft} -- \enquote{мягкие} меры (подсыпка, искусственные пляжи);
\item \textit{natural} -- \enquote{естественные} меры (восстановление экосистем, естественные мангровые леса, рифы и т.д.);
\item \textit{hybrid} -- \enquote{гибридные} меры (сочетание сооружений и экосистем).
\end{itemize}

По результатам анализа гибридные и мягкие меры в среднем показали лучшее снижение риска, показав заметно лучшие результаты относительно естественных \enquote{голых} (unvegetated) берегов.
Все рассмотренные варианты берегозащиты предполагающие человеческое участие (hard, soft, hybrid) имеют положительную экономическую отдачу в перспективе 20 лет, но мягкие и гибридные меры оказываются более рентабельны, чем чисто жёсткие сооружения.

Подтверждают эти выводы и результаты опубликованные в \cite{pinto2023coastal_defence_monitoring}.
В рамках анализа 355 кейсов из 301 научных публикаций авторы отдают предпочтение «мягким» мерам берегозащиты (подсыпка пляжей, наращивание дюн), указывая в то же время на недостаточный контроль за глобальным мониторингом прибрежной морской зоны (меньше 5\%) для контроля траектории изменения побережий в условиях происходящих климатических изменений.

Несмотря на известность мер, общие мировые тренды констатируют печальный факт того, что в борьбе за сохранение береговых линий человек проигрывает силам природы.
Так в статье \cite{Murthy2022CoastalResearch}, авторами, представляющими Министерство наук о Земле Индии, отмечается, что значительная часть пляжей в мире и на индийском побережье испытывает хроническую долгосрочную эрозию, что ведет к потере территории и деградации прибрежных экосистем.
Причину этого авторы определяют в усиливающимся антропогенном воздействии в прибрежных зонах и ускоренном повышения уровня моря вследствии глобального потепления.
Солидарны с такими выводами и авторы \cite{angnuureng_challenges_2025}, отмечающие рост масштабов береговой эрозии во всём мире и постепенный отход от «жёстких» инженерных сооружений в пользу природо-ориентированных решений.
  
Оригинальный подход к берегозащите предлагают авторы \cite{Chen2022GreenNourishment}, рассматривающие комплекс пляжной подсыпки в совокупности с высадкой морских водорослей и растений, предполагая усиление эффекта погашения волновой энергии и удержание наносов.
Моделирование показало, что наибольший эффект достигается в случае, если луг расположен в зоне прибойного вала, а намыв, в свою очередь, «экранирует» траву от разрушающего воздействия волн.
В результате предложенный метод "green nourishment" (\enquote{зелёная подсыпка}) рассматривается как перспективное \enquote{естественное} ("nature-based") решение для защиты низких песчаных берегов от эрозии и затопления, способное уменьшать штормовую опасность и одновременно поддерживать морские экосистемы.

Эффективность совмещения классической пляжной отсыпки с периодическим перераспределением пляжного материала по площади с применением бульдозеров и грейдеров предлагается в статье \cite{PELLON2023}. 
Авторы акцентируют внимание на то, что естественное накопление осадков в спокойные периоды уже не компенсирует зимнюю штормовую эрозию, вследствие чего нужны дополнительные меры против прогрессирующего размыва берега.
Рассматривают ширину пляжа как ключевой параметр для береговой защиты, туризма и других экосистемных услуг пляжей предлагаемая авторами технология "beach scraping" продемонстрировала эффективность как в лабораторных, так и в натурных условиях на примере пляжа Фуэнтебравия (Кадис, Испания).

Основываясь на рассмотренных источниках, можно утверждать, что вопросы мониторинга динамики береговой линии и управлением ее состоянием являются важными, международно востребованными и актуальными.
Мировой опыт доказывает, что пренебрежение ими создает предпосылки для аварийных ситуаций, цена ликвидации последствий которых в динамике превышает затраты на мероприятия по их предотвращению.

\subsection{Нормативное регулирование использования прибрежных территорий}
\subsubsection{Нормативные документы Российской Федерации}

Нормативное регулирование прибрежных территорий в Российской Федерации представляет собой сложную иерархическую систему, стартующую с Конституции РФ и доходящую до региональных нормативных актов и технических нормативов. 

Согласно 42 статье Конституции РФ: "Каждый имеет право на благоприятную окружающую среду, достоверную информацию о ее состоянии и на возмещение ущерба, причиненного его здоровью или имуществу экологическим правонарушением", что определяет необходимость своевременных защитных и мониторинговых мероприятий, обеспечивая тем самым статью 9: "Земля и другие природные ресурсы используются и охраняются в Российской Федерации как основа жизни и деятельности народов, проживающих на соответствующей территории".

Водный кодекс Российской Федерации \cite{WaterCodeRF} является базовым документом, регулирующим водные отношения. 
Ключевое значение имеет статья 65, определяющая понятия "водоохранных зон" и "прибрежных защитных полос", устанавливающая их размеры и режим использования. 
"Инженерная защита территорий и объектов от затопления, подтопления, разрушения берегов водных объектов, заболачивания и другого негативного воздействия вод" рассматривается в статье 67.1 этого документа, делегируя конкретику мероприятий по "строительству берегоукрепительных сооружений, дамб и других сооружений, предназначенных для защиты территорий...", области законодательства о градостроительной деятельности,
оставляя за государством право на изъятие земельных участков "в целях строительства сооружений инженерной защиты территорий и объектов от негативного воздействия вод" в порядке установленном земельным и гражданским законодательством.
Отдельно выделяется подпункт 7 этой статьи, определяющий обязательства по охране прибрежных территорий: "Собственник водного объекта обязан осуществлять меры по предотвращению негативного воздействия вод и ликвидации его последствий. 
Меры по предотвращению негативного воздействия вод и ликвидации его последствий в отношении водных объектов, находящихся в федеральной собственности, собственности субъектов Российской Федерации, собственности муниципальных образований, осуществляются исполнительными органами государственной власти или органами местного самоуправления в пределах их полномочий в соответствии со статьями 24 - 27 настоящего Кодекса".

Статья 6 ВК РФ регламентирует размеры береговой полосы вдоль водных объектов общего пользования.
В то время как порядок определения и оформления границ водоохранных зон и прибрежных защитных полос установлен "Постановлением Правительства РФ от 31.10.2024 № 1459" \cite{RF_Government_1459_2024}.
Важно отметить, что в обновленной редакции документа не конкретизируюется периодичность уточнения положения береговой линии, отсылая по этим вопросам к правилам, утверждённым "Постановлением Правительства РФ от 29.04.2016 № 377" \cite{postanovlenie_pravitelstva_rf_377_2016}.

Демаркировав границы водных объектов, дальнейшее правовое изучение вопроса следует выполнить со стороны суши.
Земельный кодекс РФ \cite{zemelnyj_kodeks_rf_2001} рассматривает в статье 13 вопросы охраны земель, обременяя собственников, землепользователей и арендаторов земельных участков необходимостью "защиты земель от водной и ветровой эрозии".

В Градостроительном кодексе \cite{gradostroitelnyj_kodeks_rf_2004} прямо не выделяются понятия "прибрежных территорий" и синонимичных по смыслу объектов, однако обширно используются общие конструкции про территории, на которых действуют "иные специальные режимы" (зоны с особыми условиями использования территорий, особо охраняемые природные территории, и т.д.).
Таким образом для прибрежных территорий, (согласно ГрК РФ) ПЗЗ закрепляют все водные и экологические ограничения (водоохранные и прибрежные полосы, зоны затопления/подтопления по статье 67.1 ВК РФ) и задают допустимые виды застройки и ее параметры. 
ГПЗУ затем прикладывает эти режимы к конкретному участку, указывая, что именно и на каких условиях там можно строить или размещать, превращая общие нормы ВК РФ, ЗК РФ и ГрК РФ в конкретные требования для проекта. 

Общие рамки методов контроля за состоянием берегозащитных сооружений определяет ГОСТ Р 59241-2020 «Берегозащитные сооружения. Правила обследования и мониторинга технического состояния» \cite{gost_r_59241_2020}.
Базовым документом, связывающим юридические требования по безопасности и охране среды с конкретными инженерными параметрами морских берегозащитных сооружений является СП 277.1325800.2016 «Сооружения морские берегозащитные. Правила проектирования» \cite{sp_277_1325800_2016}.
Этот документ устанавливает правила проектирования морских берегозащитных сооружений на открытых побережьях внутренних бесприливных морей и может применяться для берегов озёр и водохранилищ, задавая основу для инженерных решений по защите берега от размыва и волнения, развивая комплекс мер предусмотренных в \cite{sp_32_103_97}.

\subsection{Основные понятия, сущности и базовые процессы}

Обобщая изложенный ранее материал, следует признать отсутствие конкретного определения, формализирующего такое комплексное понятие как морские прибрежные территории.
Используемые в ВК РФ понятия \enquote{водоохранных зон} и \enquote{прибрежных защитных полос} преследуют цель определить пространство минимизирующее риск для самого водного объекта.
Определения из гидротехники позволяют описать береговую зону уже как комплекс состоящий из берега, подводного склона и пляжа, замкнув механические процессы происходящие между ними в одну систему.
Однако ни один из этих подходов, решая собственные, профильные задачи, не учитывает явно сценарии использования территорий прилегающих к берегу тем самым вырывая их из окружающего контекста.
Таким образом, комплексность объекта исследования требует для формулирования своего определения своей декомпозиции по основным особенностям.

Самым простым и очевидным признаком выделяющим прибрежные зоны является близость к воде, не самая глубокая мысль, но она сразу определяет несколько важных условий.
Наличие специфичного признака в расположении территории сразу ограничивает существующие площади суши подпадающие под формулируемое определение.
Следующий эпитет \enquote{морские} еще более сужает область исследования.
Исторический раздел представляемой работы позволяет утверждать, что прибрежные земли являются не только ограниченным, но и ценным ресурсом.
Специфичные качества прибрежных территорий делают их восстребованными в различных отраслях народного хозяйства, начиная от промышленного (портового) и заканчивая рекреационным и жилым назначением.
Важно, что в случае последних на область исследования накладываются еще ограничения по климату, который мог бы показаться привлекательным для приезжающих и рельефу местности, определяющему привлекательность и экономическую эффективность участков местности под жилое и инфраструктурное строительство.

В то же время противоречия в интересах землепользователей прибрежных территорий формируют неразрешимые конфликты интересов. 
При этом точкой фокуса, градус конфликта интересов в которой достигает максимального накала, являются прибрежные города.
Равновесие между сосредоточенными в них промышленными предприятиями и исторической застройкой в настоящее время нарушается возрастающими на эти территории объемами туристической нагрузки и дополнительного жилого строительства, продуцирующего больше инвестиционные и товарные функции земель.

Формируемые при ускоряющейся урбанизации прибрежных территорий процессы запускают некую автокаталитическую реакцию -- привлекательные земли становятся дефицитнее, цена их возрастает, инвестиционная привлекательность возрастает, позволяя переводит в класс привлекательных участки ранее не рассматриваемые как рентабельные.
Происходящие при этом увеличение антропогенной нагрузки на земли выводит их из баланса природного равновесия и запускает механизмы разрушения.

Основные законы РФ согласованно определяют необходимость мероприятий по поддержанию и сохранению экологических и топографических свойств всех территорий нашей Родины.
Сохранение этих свойств -- залог стабильности материального благополучия, как нашего, так последующих поколений.
В то же время, это благополучие недостижимо без разумного использования доставшихся нам территорий, что в свою очередь выводит их из природного равновесия.
При этом поиск сиюминутного соблазна экономической выгоды в моменте часто приводит к игнорированию перспективных гидромеханических и геомеханических опасностей, оставляя их разрешение будущим землепользователям.

Специфичной чертой морских прибрежных территорий является то, что процессы деградации и разрушения этих земель происходят кратно интенсивнее чем у других водных объектов.
Происходящий вследствие абразии и переноса наносов размыв берега представляет непосредственную угрозу разрушения для прилегающих к береговой линии объектов и потенциальную опасность для людей, на них находящихся.
Скорость протекания этих деструктивных процессов в основном определяется геологическим строением береговой линии и интенсивностью волновых и штормовых нагрузок. 
При этом интенсивность возникающих волновых процессов напрямую связана с доступной к волновому нагону площади водного объекта, что очевидно максимально проявляет себя на океанских и морских побережьях, что требует для своего изучения привлечения отдельных метеорологических и топографических знаний. 

Таким образом несмотря на разные взгляды на способы использования прибрежных земель, необходимость контроля и поддержания их стабильного состояния сформировала целый класс гидротехнических мероприятий по берегозащите.
Форма и назначение берегозащитных сооружений может быть различной и допускать различные сценарии их использования.
Однако разнообразие возможных мер защиты не позволяет сформировать универсального способа, отвечающего одновременно интересам всех сторон.

Фактически все берегозащитные сооружения определяет, помимо основной функции, пугающе высокая стоимость работ по их сооружению и реконструкции, зачастую недоступная к оперативному выделению в случае черезвычайной ситуации.
Экономически обоснованное разрешение таких ситуаций может быть достигнуто только при плановой и централизованной работе по планированию и проектированию объектов на прибрежных территориях, согласованной с оценкой потенциальных рисков в конкретной локации и систематическому мониторингу береговой линии.
Все эти работы сами по себе требуют выделения существенных человеческих и материальных ресурсов, рациональность привлечения которых может быть достигнута путем классификации прибрежных зон по степени опасности проявления негативных явлений.

\subsubsection{Определение \enquote{морской прибрежной территории}}

В связи с отсутствием утверженного в нормативных документах или общеиспользуемой практике термина \enquote{морские прибрежные территории} его необходимо сформулировать в рамках выполняемой работы.
Исходя из описанных ранее условий, под этим понятием в рамках выполняемой НИР будет пониматься следующие определения:

\vspace{1em}
\noindent \textit{Прибрежная территория} -- это часть земной поверхности, граничащая с водным объектом и характеризующаяся взаимным влиянием суши и водного объекта на состояние друг друга.

\vspace{1em}
\noindent \textit{Морская прибрежная территория} -- это прибрежная территория, граничащим водным объектом которой является море или океан.

\vspace{1em}
\noindent \textit{Урбанизированная прибрежная территория} -- это прибрежная терртория, находящаяся в границах населенного пункта, объекты которой состоят в технической, экономической, социально-культурной связи с водным объектом.


\subsubsection{Диаграмма сущностей}

Обобщая сказанное, морские прибрежные территории являются лакомым и желанным объектом, формирующим вокруг себя клубок противоречий интересов множества сторон.
Попытка распутать и препарировать который представлена в схеме, представленной на рисунке 
\ref{pic:ДиаграммаCущностей}.

Базовым объектом в представленной диаграмме является \enquote{Объект} -- это абстрактная сущность обладающая несколькими необходимыми и имманентными в своих реализациях атрибутами: состоянием, стоимостью, пользователем, назначением и степенью потенциальной опасности (риском).
В конкретной реализации объект может являться контейнером для других объектов, становясь при этом комплексным.
Базовые атрибуты объекта в целом говорят сами за себя, но понятие \enquote{Пользователя} следует рассмотреть подробнее.
Одним из методов \enquote{Объекта} является общий метод \enquote{приносить пользу}, реализуемый в потомках объекта за счет полиморфизма, выгодоприобретателем при этом становится \enquote{Пользователь}, аккумулирующий ее в себе в виде условных \enquote{денег}, имеющий возможность ими распоряжаться, но и нести ответственность за относящиеся к нему объекты.
Специфичные способы использования реализуются через спецификацию \enquote{Назначения} объекта, но универсальными являются процессы разрушения объекта (снижения собственного состояния), методы контроля состояния и методы его восстановления, требующие для своего выполнения времени и денег, что обеспечивается пользователем объекта.
Стремление к сбалансированной минимизации этих параметров достигается за счет возможности использования специфичных методов контроля и восстановления реализуемых для каждого типа объектов индивидуально.

% \begin{landscape}
% \begin{figure}[!htbp]
%     \centering
%     \makebox[0pt][c]{
%         \includegraphics[width=1.15\paperwidth,keepaspectratio]{fig/Диаграмма_сущностей}
%     }
%     \caption{Диаграмма сущностей системы}
%     \label{pic:ДиаграммаCущностей}
% \end{figure}
% \end{landscape}

Рассматриваемая на схеме сущность \enquote{Территория} является реализацией класса \enquote{Объект}, дополняя его собственными специфичными характеристиками (рельефом, климатом, запасами полезных ископаемых и т.п.) и отдельно выделяемым \enquote{Типом территории}.
Имплементируемый \enquote{Тип территории} позволяет стандартизировать территории по комплексу из двух признаков: типу \enquote{Локации} и \enquote{Способу Защиты} территории, необходимость которого следует по закону.
Сущность \enquote{Локации} является абстрактной, но позволяет классифицировать территории по собственным специфичным признакам.
Так в рассматриваемом примере они разделены на \enquote{Прибрежные} и \enquote{Материковые}.
Ключевой особенностью \enquote{Прибрежного} типа является динамичность береговой линии, механизм протекания и скорость которой определяется принадлежностью к конкретному типу водного объекта.

Наравне с \enquote{Локацией} необходимым для \enquote{Типа Территории} является и ее \enquote{Способ Защиты}.
Для всех объектов реализующую эту сущность свойственно наличие таких атрибутов как цена (стоимость работ по организации защиты), ее эффективность (определяющая темп скорости снижения \enquote{состояния} территории и включенных в нее \enquote{объектов}), срок службы и опциональные дополнительные функции и включаемые \enquote{объекты}.
Базовым поведением сущности является \enquote{защищать территорию}, предотвращая ее износ.

Примерами реализации \enquote{Способов Защиты} в контексте выполняемой НИР являются \enquote{Гидротехнические} способы защиты, классификация которых выполнена на основе работы \cite{Huynh2024HybridCoastalDefence}.
Примером \enquote{дополнительных функций} некоторых из типов берегозащиты может являться рекреационное назначение или возможность размещения морских транспортных средств.

Раздробленность  между сущностями разнородных качеств определяет сложность правильного выбора конкретных объектов размещяемых на прибрежных территориях и способов их защиты.
Сущностью принимающей на себя сложность выбора последних ложится на предлагаемый к формированию в работе новый тип объекта \enquote{Фабрика Способов Защиты}.
Механизм выбора конкретного способа может быть реализован в нем различными способами, в зависимости от контекста решаемой задачи, но необходимым и фундаментальным условием определяющем систему принимаемых решений должен стать общий для всех способов источник \enquote{информации о смежных территориях}.
Этот источник должен агрегировать в себе не только информацию конкретной локации, доступную самой территории в ее атрибуте \enquote{характеристик}, но и прилегающим и содержащихся в них \enquote{объектах}.
Наличие такой информации позволит учесть интересы всех \enquote{пользователей} в принимаемом решении и достигнуть максимальной эффективности.

Очевидно, что глобальность необходимого агрегатора информации не позволяет сформировать его в рамках одного НИР, но ограничиваясь конкретными \enquote{морскими прибрежными территориями} возможно  описать локализацию скорость протекания деформационных на них процессов, определив тем самым наиболее перспективные к изучению и освоению территории.   


%%%%%%%%%%%%%%%%%%%%%%%%%%%%%%%%%%%%%%%%%%%%%%%%%%%%%%%%%%%%%%%%%%%

\subsubsection{Диаграмма процесса разработки генерального плана морских прибрежных территорий с учетом влияния водного объекта}

В связи с требованиями защиты прибрежных земель от разрушения, устанавливаемыми ВК РФ и ЗК РФ, организация необходимых мероприятия по их организации должна быть учтена при разработке или корректировке существующих генеральных планов морских прибрежных территорий, что является стартовой точкой диаграммы, изображенной на рисунке \ref{pic:ДиаграммаПроцесса}. 

% \begin{landscape}
% \begin{figure}[!htbp]
%     \centering
%     \makebox[0pt][c]{
%         \includegraphics[width=1.15\paperwidth,keepaspectratio]{fig/Диаграмма_процесса_НИР_1}
%     }
%     \caption{Диаграмма процесса разработки генерального плана морских прибрежных территорий с учетом влияния водного объекта}
%     \label{pic:ДиаграммаПроцесса}
% \end{figure}
% \end{landscape}

Для принятия рациональных решений при этом необходимо классифицировать береговую линию согласно вероятности проявления негативных, разрушительных процессов, вызванных влиянием моря на прибрежную территорию. 
Анализ научных источников проведенный в разделе \ref{sec:cvi} показал, что наиболее перспективным и изученным в мире параметром для описания такой вероятности является индекс уязвимости береговой зоны (Coastal Vulnerability Index (CVI)).
К сожалению, практика использования CVI в городском планировании и анализе к настоящему моменту не выполняется, хотя опыт многих стран доказывает эффективность его применения в течение последних 30 лет.

Одним из препятствий к использованию CVI является то, что для его определения необходимы специфичные исходные данные, описывающие граничащий с территорией водный объект.
Спецификация этих данных и варианты их сбора описаны в разделе \ref{sec:data} данной работы.
Определив CVI, становится возможным классифицировать участки берега по вероятности проявления на них негативных процессов разрушения береговой линии или затопления прилегающих к ней территорий.

В позитивном случае (когда вероятность наступления этих событий принимается допустимой) в работе предлагается выполнение процесса планирования территории согласно действующим методиками, с единственным дополнительным обременением, заключающимся в прописывании регламента мероприятий по мониторингу состояния береговой линии и ее восстановлению при необходимости.
Следует отметить, что предлагаемое обременение в работе предлагается использовать во всех случаях использования прибрежной территории, гарантируя тем самым контроль исполнения федеральных законов и кодексов.

Негативный сценарий анализа предполагает реакцию на наличие на рассматриваемой территории участков с высокой вероятностью негативных процессов.
В этом случае необходим дополнительный анализ исходных данных примененных в расчете CVI.
В случае сомнений в качестве использованных показателей целесообразно их уточнение путем дополнительных инженерных геодезических, батиметрических, геологических и метеорологических изысканий.
Уточнив в результате получения новых данных CVI, становится возможным точнее локализовать участки берега с неприемлемой вероятностью негативных явлений и определить границы конкретной зоны влияния водного объекта.
Ход дальнейшего процесса планирования рекомендуется выполнять с учетом типа функциональной зоны, попадающей в зону вредного влияния водного объекта.

Так, размещение на этих участках промышленных зон будет являться для города более предпочтительным, так как все мероприятия по гидрозащите лягут на плечи землепользователя.
С учетом того, что размещаемые на этих землях промышленные объекты (порты, нефтебазы, перевалочные комплексы, доки и т.п.) имеют высокий класс опасности и подпадают под контроль Ростехнадзора, мероприятия по проектированию, строительству и эксплуатации гидротехнических сооружений организуются пользователем этих объектов.
Дополнив сказанное тем, что все разрабатываемые проекты для этих объектов проходят государственную экспертизу, дополнительные решения на этапе составления генерального плана для них не требуются.

Наибольший интерес в области принятия решений по использованию прибрежных территорий остается локализован на участках жилого и рекреационного испльзования, попадающих в зону с высокой вероятностью негативных процессов вдоль берега.
В области пересечения этих зон необходим дополнительный анализ риска, выполненный с учетом специфики существующих и планируемых к размещению объектов.
В результате учет полученных показателей может являться критерием для выбора способа гидрозащиты территории, учитывающей как необходимые средства для их организации так и эксплуатации.


\begin{landscape}
\begin{figure}[!htbp]
    \centering
    \makebox[0pt][c]{
        \includegraphics[width=1.15\paperwidth,keepaspectratio]{fig/Диаграмма_сущностей}
    }
    \caption{-- Диаграмма сущностей системы}
    \label{pic:ДиаграммаCущностей}
\end{figure}
\end{landscape}

\begin{landscape}
\begin{figure}[!htbp]
    \centering
    \makebox[0pt][c]{
        \includegraphics[width=1.15\paperwidth,keepaspectratio]{fig/Диаграмма_процесса_НИР_1}
    }
    \caption{-- Диаграмма процесса разработки генерального плана морских прибрежных территорий с учетом влияния водного объекта}
    \label{pic:ДиаграммаПроцесса}
\end{figure}
\end{landscape}



\newpage
\section{Методология решения проблемы}

\subsection{Индекс уязвимости береговой зоны (CVI)}\label{sec:cvi}

Coastal Vulnerability Index (CVI) (Индекс уязвимости береговой зоны) -- один из наиболее распространенных и методологически выверенных подходов к количественной оценке уязвимости морских побережий. 
История его возникновения неразрывно связана с ростом научного интереса к проблемам изменения климата и повышения уровня моря, произошедшим в начале 1990-х годов. 
К настоящему времени CVI прошел путь от экспериментальной методики NASA до глобального стандарта в управлении прибрежными зонами.

Основоположником метода определения CVI считается Вивьен Горниц (Vi\-vien Gornitz) и ее коллеги из Института космических исследований Годдарда (NASA Goddard Institute for Space Studies). 
В 1990–1991 годах на фоне первых докладов IPCC и растущей обеспокоенности глобальным потеплением, перед учеными встала задача создать инструмент для оценки риска затопления и эрозии побережья США. 
В рамках решения этой задачи Горниц предложила алгоритм, который объединял разрозненные физические параметры в единый числовой показатель уязвимости.
Первая версия CVI была протестирована на Восточном побережье США и заложила концептуальную основу для всех будущих модификаций. \cite{go02000x}.

В конце 1990-х годов Геологическая служба США (USGS) инициировала Национальную оценку уязвимости побережья, в ходе которой Роберт Тилер (Robert Thieler) и Эрика Хаммар-Клозе (Erika Hammar-Klose) доработали методику Горниц, создав то, что сегодня считается «классическим CVI» \cite{ThielerHammarKlose1999}.

Они зафиксировали список из 6 ключевых факторов, определяющих вероятность возникновения негативных деформационных процессов, и выразили формулу расчета \eqref{eq:CVI}, которая используется в большинстве исследований до нашего времени:

\begin{equation}\label{eq:CVI}
CVI= \sqrt{\frac{a \cdot b \cdot c \cdot d \cdot e \cdot f}{6}},
\end{equation}

\noindent где переменные $a$–$f$ соответствуют ранжированным по баллам (от 1 (очень низкая) до 5 (очень высокая)) оценкам для факторов:

\begin{itemize}
\item $a$ -- геоморфология берега (скалы — 1, песчаный пляж — 5), определяет естественную устойчивость берега;
\item $b$ -- скорость изменения береговой линии (м/год, из DSAS-анализа), характеризует тренды эрозии и аккреции берега;
\item $c$ -- уклон берега (крутой — низкая уязвимость, пологий — высокая), определяет потенциал к затоплению;
\item $d$ -- скорость относительного повышения уровня моря (мм/год), определяющая глобальные климатические и тектонические изменения;
\item $e$ -- средняя значительная высота волны (м), определяющая воздействие волновой энергии на берег;
\item $f$ -- средняя величина прилива (м), характеризующая приливно-отливную динамику.
\end{itemize}

Полученные значения CVI разбиваются на квартили для определения уровня уязвимости прибрежной территории (низкая, умеренная, высокая, очень высокая) \cite{Tarigan2024CVI}.

По мере распространения и интеграции ГИС-технологий и формирования концепций устойчивого развития урбанизируемых территорий, методика CVI начала трансформироваться и дополняться факторами, расширяя формулу \eqref{eq:CVI} по аналогии с уже введенными параметрами.
В ответ на распространенную критику классического CVI за игнорирование человеческого фактора появились модифицированные индексы,  которые начали включать плотность населения, наличие инфраструктуры и экономическую ценность земель \cite{McLaughlinCooper2010, SzlafszteinSterr2007}.
Выведенный в 2025 году \enquote{геотехнический} Geotechnical Coastal Vulnerability Index (GCVI), включил в себя свойства грунтов и гранулометрический состав наносов, показав свою эффективность на примере Патрасского залива в Греции \cite{Boumpoulis2025GCVI}.

Уточнение индекса под конкретные локальные территории сформировало отдельные региональные модификации в виде индексов территориальной уязвимости побережья  Coastal Territorial Vulnerability Index (CTVI).
В работе \cite{Barros2022CTVI} была предложена методология адаптированная под условия трех районов Португалии, по которым имелись исторические данные о динамике прибрежных территорий. 
Предложенный CTVI включал 33 переменные, сгруппированные в четыре группы:

\begin{itemize}
\item морфологический компонент ($Mv$) -- 6 переменных​ (тип берега, уклон, ширина пляжа, защита береговой линии, геология пород, характеристики прибрежных дюн);
\item стоимость земли ($Lv$) -- 2 переменные (Тип землепользования, коэффициенты налога на имущество);
\item здания ($Bv$) -- 10 переменных: (конструкционные материалы, этажность, состояние объекта, наличие подземных этажей, ориентация относительно побережья, гидродинамические характеристики первого этажа, тип использования, число жителей (посетителей), наличие критически важных элементов);
\item общественные пространства ($PAv$) -- 15 переменных (число пользователей, временная динамика занятости, подвижные объекты, критическая инфраструктура, садово-парковое оборудование и пр.).
\end{itemize}

На основе многокритериального анализа были вычислены веса для каждой переменной после чего CTVI рассчитывается как сумма по четырем категориям \eqref{eq:CTVI}:

\begin{equation}\label{eq:CTVI}
CTVI = Mv + Lv + Bv + PAv,
\end{equation}

\noindent где каждый компонент нормализован в диапазоне от 0 до 1.

Итоговый CTVI принимает значения от 0 (минимальная уязвимость) до 4 (максимальная уязвимость). 
Результаты расчета валидировались на основе исторических данных о  650 случаях прибрежного затопления и переливов за период 1980-2018 годов, при этом 83,3\% случаев произошли в зонах, классифицированных как имеющие умеренную, высокую и очень высокую уязвимость по CTVI.

В целом, наиболее эффективные методы оценки в той или иной степени используют внутреннее ранжирование факторов, используя как субъективные (экспертные), так и объективные (статистические) подходы для определения весовых коэффициентов.
В упомянутой ранее работе \cite{Boumpoulis2025GCVI} для этого использовался метод главных компонент (PCA), но наиболее перспективным считается энтропийный метод, основанный на вариабельности данных в котором параметры с высокой информационной полезностью (низкая энтропия) получают больший вес.
Так в работе  \cite{Fu2022HainanErosionCVI} были объединены метод Дельфи и энтропийный метод для оценки коралловых рифов Хайнаня (Китай) в результате динамические параметры (скорость эрозии береговой линии, изменение изобат) получили наивысшие веса, что было подтверждено корреляционным анализом.

Анализ вариативности параметров в расчете может быть учтен при стохастическом подходе Probabilistic Coastal Vulnerability Index (PCVI).
Так в работе \cite{Tanim2023PCVI} на анализе совместных распределений биофизических и социально-экономических факторов уязвимости было выполнено сравнение с традиционными детерминированными аддитивным и мультипликативным индексами (ACVI и MCVI) на примере приморских округов Южной Каролины. 
Результаты работы показывают, что PCVI лучше сохраняет многомерную информацию об уязвимости и точнее объясняет наблюдавшиеся последствия ураганов Florence (2018) и Matthew (2016) по данным послештормовых карт затопления и стоимости ликвидаций их последствий.

Итогом обзора CVI может являться исследование \cite{ElKotby2026CVIReview}, представляющее собой обзор 35 работ по применению индекса прибрежной уязвимости (CVI) для оценки рисков прибрежных зон, в котором автор систематизирует и анализирует преимущества и ограничения индексов CVI, а также показывает переход от эмпирических и ГИС-подходов к AI-ориентированным и климат-адаптивным схемам с применением методов машинного обучения для повышения объективности и прогностической точности. 
Ключевой вывод работы состоит в том, что «универсального» оптимального метода оценки уязвимости побережья не существует: пригодность той или иной схемы CVI определяется доступностью данных, а также конкретным контекстом применения индекса.

\subsubsection{Применение CVI  для прибрежных территорий \mbox{Российской Федерации}}

Несмотря на известность и изученность метода оценки прибрежных территорий по риску их уязвимости, выполнение расчетов для территории РФ носит локальный и фрагментарный характер.
На сегодняшний день не существует единого источника информации, объединяющего в себе как результаты расчета CVI так и исходных данных для его нахождения.
Принципиальной проблемой использования результатов локальных расчетов является то, что выполняемая классификация результатов проводится только относительно имеющегося набора значений, отчего агрегирование между различными дискретными работами невозможно в силу различной вариабельности рассчитанного индекса.

Ограниченное количество отечественных работ позволяет привести здесь фактически полный корпус источников.

Одна из наиболее показательных работ, описывающих расчет и анализ CVI для территории Куршской косы в Калининградской области, позволила локализовать наиболее опасные участки берега \cite{sukmanova2023cvi}.
Полученные в работе результаты (рисунок \ref{pic:CVI_Калининград}) показали корреляцию с известными локациями негативных гидромеханических процессов, локализующихся в районе г. Светлогорск и г. Балтийск  (рисунок \ref{pic:Динамика_берега_Калиниград}) \cite{Ryabchuk2021RussianBalticCoasts}. 

\begin{figure}
    \begin{center}
	\includegraphics[width=150mm]{CVI_Калининград.png}
	\caption{-- Степень уязвимости (методом CVI) побережья Калининградской области \cite{sukmanova2023cvi}}
	\label{pic:CVI_Калининград}
    \end{center}
\end{figure}

\begin{figure}
    \begin{center}
	\includegraphics[width=150mm]{Динамика_берега_Калиниград.png}
	\caption{-- Схема динамики побережья Самбийского полуострова в Калининградской области \cite{Ryabchuk2021RussianBalticCoasts}}
	\label{pic:Динамика_берега_Калиниград}
    \end{center}
\end{figure}

Анализ прибрежных территорий Финского залива был выполнен в работе \cite{Kovaleva2022EGoF_CVI}, результаты расчетов которой представлены на рисунке \ref{pic:CVI_Балтийский_залив}.
Экспертное взвешивание параметров, выполненное в работе показало ключевую роль двух параметров — прибрежной геологии и частоты штормов, а также значимость степени изрезанности/закрытости береговой линии -- именно они в наибольшей степени определяют запуск и развитие абразионных процессов.
В выводах работы предлагается использовать полученные результаты при планировании строительства портов, зон рекреации и других форм освоения территории, а также выделения приоритетных зон, где необходимы меры гидротехничекой защиты и ограничение техногенной нагрузки.

\begin{figure}
    \begin{center}
	\includegraphics[width=150mm]{CVI_Балтийский_залив.png}
	\caption{-- Взвешенный CVI для прибрежных территорий Финского залива \cite{Kovaleva2022EGoF_CVI}}
	\label{pic:CVI_Балтийский_залив}
    \end{center}
\end{figure}


Анализ территорий Невской губы также был выполнен в рамках диссертационного исследования \cite{Lednova2021NevskayaGuba}, в котором проведена оценка пространственно-временного распределения воздействия на экосистему, с особым акцентом на район аванпорта «Бронка» и прилегающую акваторию.

Для других водные бассейны РФ (Черное море \cite{Gogoberidze2022CriterionStatistical}, Каспийское море, Арктика \cite{Gogoberidze2025MurmanskRisks}, моря Дальнего Востока) CVI чаще фигурирует как методический референс, чем как самостоятельный, детально опубликованный расчет.

Расчет CVI для прибрежной зоны Черного моря был выполнен в работе \cite{Tatui2019BlackSeaErosion} с дискретизацией в 1 км.
По результатам опубликованной работы четверть всего побережья Черного моря отнесена к наиболее чувствительным  и опасным к эрозии и затоплению участкам, в основном это низменные песчаные берега и области дельт втекающих рек.
Для этих зон авторами были предложены приоритетные управленческие меры по ограничению застройки, адаптация инфраструктуры, целенаправленные берегозащитные мероприятия и усиленный мониторинг.
Несмотря на качество выполненной работы мелкий масштаб приведенных в работе иллюстраций (рисунок \ref{pic:CVI_Черное_море}) не позволяют локализовать конкретное положение опасных участков, а приведенное на иллюстрациях распределение пропорций зон не соответствует конституционному положению границ РФ, а также отсутствуют расчеты для берегов Азовского моря.
Все это позволяет лишь проиллюстрировать и отметить фактическое наличие проблемных зон на прибрежной территории Краснодарского края, республики Крым и Херсонской области.

\begin{figure}
    \begin{center}
	\includegraphics[width=150mm]{CVI_Черное_море.png}
	\caption{-- Результаты расчета CVI для Черного моря~\cite{Tatui2019BlackSeaErosion}}
	\label{pic:CVI_Черное_море}
    \end{center}
\end{figure}

\subsubsection{Выводы}

Давая характеристику CVI, как показателю позволяющему оценить устойчивость береговой линии, следует признать его эффективность и удобство в качестве метрики, характеризующей вероятность проявления на территории деструктивных гидромеханических процессов.

В то же время анализ работ по его использованию для морских прибрежных территорий РФ показал ограниченное использование как в научной, нормативной так и методической литературе.
Имеющийся опыт применения (относящийся к прибрежным территориям Черного и Балтийского морей) не охватывает всю географию нашей страны, а несогласованность показателей между собой не позволяет выполнить объединение их результатов в единой шкале.

Последнее утверждение в целом относится ко всему корпусу работ по этой теме.
Естественная простота и наглядность формулы \eqref{eq:CVI}, по которой вычисляется CVI позволяет легко ее модифицировать путем добавления или исключения новых признаков, а использование ранжированных показателей -- игнорировать проблемы их размерностей и вариации.
Цена такой свободы проявила себя в том, что фактически каждое отдельное исследование находится в своей собственной, уникальной шкале, используемой только внутри отдельной, рассматриваемой территории.
В связи с этим, зоны, принимаемые как умеренно опасными относительно своих соседей, были бы безопасными в других местах, обладающих большей динамикой изменения береговой линии.
Снизить этот эффект можно в ходе совместного анализа всей прибрежной территории, что, очевидно, потребует формирования единой методики сбора и обработки пространственных данных. 

\subsection{Идея нового предлагаемого в НИР подхода}

Основываясь на выводах предыдущего раздела, можно сформулировать следующую генеральную задачу -- разработка единого метода для формирования возможности вычисления CVI одновременно на всей территории РФ.
Условием для существования этого метода является возможность получения полного набора необходимых параметров для вычисления CVI в любой точке побережья, что может быть достигнуто путем формирования единой базы данных их содержащих или стабильных методов их получения и расчета в режиме реального времени.
Опыт проанализированных работ показал, что часть необходимых параметров может быть рассчитана на основе данных находящихся в открытом доступе или доступном в режиме ограниченного доступа.
Проблемой при этом остается их разрозненность и разнородность, требующая от исследователя больших затрат времени и сил на их агрегацию и предварительную обработку.

Разработка единого интерфейса совмещающего инструменты получения, обработки и вычисления результирующего CVI позволили бы стандартизировать эти процедуры.
Очевидным недостатком такого подхода, в первое время, стала бы ограниченность ряда используемых параметров доступных к обработке, но наличие единого \enquote{фундамента} позволило бы перенаправить высвободившиеся силы исследователей на согласованное улучшение формируемой базы делающего работы в этой области более последовательными, надежными  и верифицируемыми.




\newpage
\section{Исходные данные}

\subsection{Спецификация исходных данных}\label{sec:data}

Для осуществления ранее сформулированной идеи по созданию единого агрегатора данных, необходимых для вычисления CVI индекса в произвольных точках прибрежных территорий РФ, необходимо определить исходные данные необходимые для вычисления CVI.
Анализ изученных источников позволяет выделить следующие из них:

\begin{itemize}
\item положение береговой линии;
\item геоморфология рельефа; 
\item параметры волн;
\item скорость подъема уровня моря;
\item скорость изменения береговой линии.
\end{itemize}

\subsubsection{Положение береговой линии}

Положение рассматриваемой береговой линии позволяет не только определить координаты для определения последующих параметров, но и определить ориентацию берега, то насколько он открыт к преобладающему волнению и штормовым направлениям ветра, а значит, к волновой энергии, разрушающей берег и находящиеся на нем объекты.

Так как положение береговой линии определяется простым линейным объектом получить его можно путем стандартных запросов из OSM. 
Ключевой тег для запроса береговой линии океанов и морей в нем: \verb|natural=coastline|.

Одной из проблем при работе с извлеченной береговой линией будет являться большое количество составляющих ее вершин, что может быть купировано путем ее разделения на набор последовательных фрагментов равной длины, позволяющие, в дальнейшем, локализовать вычисления внутри соответствующих им тайлов.
Выполнить такое разделение можно путем использования функции \verb|split| из пакета \verb|shapely.ops|.

\subsubsection{Геоморфология рельефа}

Морфометрию рельефа, необходимую для вычисления CVI можно ограничить его экспозицией: наклоном, кривизной, ориентацией в пространстве.
Следует отметить, что для качественного вычисления этих параметров вдоль линии берега обычные топографические данные описывающие рельеф необходимо дополнить еще и батиметрическими данными, описывающими рельеф прилегающего дна.
Стандартным форматом для представления таких данных являются DEM модели (растровый набор данных, где каждый пиксель содержит высоту земной поверхности над уровнем моря).
Бесплатные источники таких данных доступны из нескольких глобальных проектов, причем большинство из них имеют открытые API для работы.

Получение информации о рельефе на суше доступно из проектов:

\begin{itemize}
\item SRTM (Shuttle Radar Topography Mission);
\item ASTER GDEM (Advanced Spaceborne Thermal Emission and Reflection Radi\-ometer);
\item ALOS World 3D (AW3D30);
\item FABDEM — Forest And Buildings DEM.
\end{itemize}

Во всех случаях пространственное разрешение моделей высот составляет порядка 30 м (1" по широте и долготе), что вполне достаточно для вычисления искомых характеристик.

Доступные глобальные батиметрические модели обладают более скромными параметрами и доступны из проекта GEBCO (General Bathymetric Chart of the Oceans) с пространственным разрешением 15".

\subsubsection{Параметры волн}

В большинстве работ по глобальной оценке волн на прибрежные территории используются глобальные источники данных:

\begin{itemize}
\item NOAA Wave Watch III (WW3) с разрешением 0.5° (примерно 55 км на экваторе);
\item ECMWF ERA5 Reanalysis (Copernicus) с разрешением 0.25° (примерно 28 км).
\end{itemize}

Однако опыт рассмотренных работ показал, что использовать данные с такой низкой плотностью ограниченно полезно.
К счастью методики расчета параметров волнового наката являются хорошо изученными и методически описанными в гидротехнике.
Пример такой методики утвержденной в РФ описан в СП 38.13330.2018 \enquote{Нагрузки и воздействия на гидротехнические сооружения (волновые, ледовые и от судов)} \cite{SP38_13330_2018}.

Исходными данными к выполняемым расчетам по данной методике являются батиметрия, положение береговой линии и усредненные диаграммы по розе ветров.
Если первые два параметра уже упоминались ранее как необходимые источники информации параметры ветров и погоды следует рассмотреть отдельно.

\vspace{1em}
\textbf{Получение метеорологических данных}
\vspace{1em}

В Российской Федерации основным ответственным органом за хранение метеорологических данных является Федеральная служба по гидрометеорологии и мониторингу окружающей среды (Росгидромет), которая формирует и ведет Единый государственный фонд данных о состоянии окружающей среды (ЕГФД).
Формально информация ЕГФД является открытой и общедоступной, за исключением сведений ограниченного доступа, но свободный прямой онлайн-доступ к полным архивным метеоданным отсутствует.

Наравне с этим в мире существует набор агрегаторов метео данных, предоставляющих доступ к своей информации через удобные API.
В рамках НИР были использованы 4 следующих открытых источника метеоданных:

\begin{itemize}
\item Open-Meteo -- данные включают исторические архивы (с 1940 г.) по следующим параметрам: температура, осадки, ветер, давление, облачность, UV-индекс и др. для 10 км пространственного разрешения;
\item Meteostat;
\item Visual Crossing Weather;
\item OpenWe\-atherMap.
\end{itemize}

Во всех сервисах присутствует возможность получения исторических данных о погоде, однако они ограниченны в бесплатном доступе (как в OpenWea\-ther\-Map) или фрагментарны (как в Meteostat).
Однако Meteostat прекрасно справляется с выдачей исторической статистики о работе метеостанциий по территориям во времени, историческое положение метеостанций может быть использовано при запросах в Visual Crossing Weather, позволяющему получить наиболее полную информацию, но ограниченное количество раз в сутки.

Наиболее перспективным сервисом же представляется Open-meteo, допускающий большее число дневных запросов, но выдающий данные только внутри растровой сетки с пространственным разрешением в 9 км.

Для автоматизации получения и сохранения данных был создан проект "Be\-ach\-Wea\-ther", перейти к которому можно по ссылке (рисунок \ref{pic:QR_code_BW}):

\begin{figure}
    \begin{center}
	\includegraphics[width=40mm]{fig/QR_code_BW.png}
	\caption{-- QR code на проект "BeachWeather"}
	\label{pic:QR_code_BW}
    \end{center}
\end{figure}

На текущий момент в нем реализованы базовые скрипты получения данных с упомянутых выше сервисов в формате  \verb|JSON| и их парсинга для сохранения в базе данных. 
Также была сформирована база данных (рисунок \ref{pic:БД_погода}) позволяющая накапливать в себе информацию для аккумуляции информации и демпфирования ограничений в доступных запросах.

\begin{figure}
    \begin{center}
	\includegraphics[width=\textwidth]{fig/БД_погода.png}
	\caption{-- Схема базы данных для агрегации метеорологических данных}
	\label{pic:БД_погода}
    \end{center}
\end{figure}

Структура разработанной базы данных для накопления метеорологических данных

\subsubsection{Скорость подъема уровня моря}

Скорость подъема уровня моря для расчета CVI можно получить из нескольких открытых источников:

\begin{itemize}
\item Permanent Service for Mean Sea Level — PSMSL;
\item Copernicus Marine Service (CMS);
\item ESA Climate Change Initiative (CCI).
\end{itemize}

PSMSL, предоставляющий данные об измерении уровня моря (мареографические данные) -- это крупнейшая глобальная база данных уровня моря с береговых мареографов.
Этот источник данных позволят проанализировать их на большем временном промежутке, однако сильно привязан к конкретным точкам положения наблюдательных станций.

Данные о глобальной альтиметрии уровня вод с большим пространственным разрешением доступны через сервисы CMS и CCI, но содержат данные с 1993 года, так как используют данные космического зондирования.
Пространственное разрешение этих данных 0.25°, что, с учетом малой скорости измерений, вполне достаточно для анализа. 

\subsubsection{Cкорость изменения береговой линии}

Параметры описывающие скорость изменения береговой линии являются наиболее труднодоступными и наименее надежными для анализа.
Наилучшие результаты достигаются при периодическом мониторинге геодезическими и фотограмметрическими методами, но они применимы только на локальных участках.

Альтернативой, при глобальном анализе, могут являться данные полученные при обработке радарных космических снимков.
Примеры выполненных на их основе расчетов сформированы в базах USGS National Shoreline Change (для побережий США) и European Marine Observation and Data Network (для территорий Европы).
Прибрежные территории РФ требуют отдельного расчета по методикам DSAS (Система цифрового анализа береговой линии), реализованной в пакете coastsat для python, однако необходимость обработки большого массива последовательных космических снимков систем Landsat 5/7/8/9 и Sentinel-2 скачивание которых затруднено в связи с санкционным режимом наложенным на РФ делает эту задачу труднореализуемой.

\subsection{Выводы по разделу}

Анализ доступных источников открытых данных показал, что есть все основания полагать, что поставленная в работе цель может быть достигнута на их основе.
Приведенный в работе обзор потенциальных источников требует дальнейшей проработки и подготовки специальных интерфейсов для своего получения, чему будет уделено основное внимание во время подготовки второго этапа НИР.


\newpage
% \section*{ЗАКЛЮЧЕНИЕ}
\begin{center}
  \textbf{ЗАКЛЮЧЕНИЕ}
\end{center}
\addcontentsline{toc}{section}{ЗАКЛЮЧЕНИЕ}

Выполненный в ходе 1 этапа НИР анализ предметной области исследований показал, что вопрос эффективного, экологичного и безопасного использования прибрежных морских территорий является востребованным и актуальным по всему миру.
Множество научных публикаций по данному вопросу в высококвартильных изданиях, международных конференций и исследований подтверждает тезис, что проблема разрушения и деградации прибрежных полос наблюдается по всему миру и до сих пор не имеет однозначного решения.

Проведенный анализ истории взаимодействия человека с объектом исследования позволяет утверждать, что несмотря на объективные риски освоения прибрежных земель, блага от такого соседства позволяют компенсировать затраты как на научное сопровождение работ по изучению процессов происходящих на прибрежных территориях, так и мероприятия по овеществлению сформулированных в их ходе рекомендаций.
Научный поиск позволяет также утверждать, что мероприятия организованные по защите и укреплению береговой линии в перспективе окупаются, чего нельзя сказать про бездействие в этом вопросе.

Анализ методов описания вероятности проявления негативных гидродинамических процессов вдоль берега показал, что наиболее изученным и универсальным показателем для его описания является индекс уязвимости береговой зоны (Coastal Vulnerability Index (CVI).
Международный опыт использования этого показателя доказал его применимость в вопросах территориального планирования объектов вдоль береговой линии морей и океанов.
Однако, несмотря на большое количество публикаций по развитию и применению CVI опыт его применения для прибрежных территорий Российской Федерации ограничивается всего парой исследований.
Объективным препятствием к исправлению ситуации в этом вопросе является отсутствие прямого доступа к подготовленным данным, необходимым для его расчета.

Выполненный в третьей части работы анализ открытых источников глобальных данных позволяет надеяться на возможность формулирования методики их агрегации для вычисления показателя CVI на требуемой для территориального планирования территории, усилия на что и будут направлены в рамках следующего этапа НИР.
Сформированные при этом методы получения и обработки геопространсвенных данных могут быть применимы в решении научных и инженерных задач применительно и к прочим территориям.



% Не редактируем: Страница библиографии (формируется автоматически из книжек, указанных в refs.bib и пометок \cite{имя_источника} в тексте)
\newpage
\begin{center}
  \textbf{СПИСОК ИСПОЛЬЗОВАННЫХ ИСТОЧНИКОВ}
\end{center}
\addcontentsline{toc}{section}{СПИСОК ИСПОЛЬЗОВАННЫХ ИСТОЧНИКОВ}
\renewcommand{\refname}{}  

\printbibliography

\end{document}
