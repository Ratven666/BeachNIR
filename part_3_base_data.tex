
\newpage
\section{Исходные данные}

\subsection{Спецификация исходных данных}\label{sec:data}

Для осуществления ранее сформулированной идеи по созданию единого агрегатора данных, необходимых для вычисления CVI индекса в произвольных точках прибрежных территорий РФ, необходимо определить исходные данные необходимые для вычисления CVI.
Анализ изученных источников позволяет выделить следующие из них:

\begin{itemize}
\item положение береговой линии;
\item геоморфология рельефа; 
\item параметры волн;
\item скорость подъема уровня моря;
\item скорость изменения береговой линии.
\end{itemize}

\subsubsection{Положение береговой линии}

Положение рассматриваемой береговой линии позволяет не только определить координаты для определения последующих параметров, но и определить ориентацию берега, то насколько он открыт к преобладающему волнению и штормовым направлениям ветра, а значит, к волновой энергии, разрушающей берег и находящиеся на нем объекты.

Так как положение береговой линии определяется простым линейным объектом получить его можно путем стандартных запросов из OSM. 
Ключевой тег для запроса береговой линии океанов и морей в нем: \verb|natural=coastline|.

Одной из проблем при работе с извлеченной береговой линией будет являться большое количество составляющих ее вершин, что может быть купировано путем ее разделения на набор последовательных фрагментов равной длины, позволяющие, в дальнейшем, локализовать вычисления внутри соответствующих им тайлов.
Выполнить такое разделение можно путем использования функции \verb|split| из пакета \verb|shapely.ops|.

\subsubsection{Геоморфология рельефа}

Морфометрию рельефа, необходимую для вычисления CVI можно ограничить его экспозицией: наклоном, кривизной, ориентацией в пространстве.
Следует отметить, что для качественного вычисления этих параметров вдоль линии берега обычные топографические данные описывающие рельеф необходимо дополнить еще и батиметрическими данными, описывающими рельеф прилегающего дна.
Стандартным форматом для представления таких данных являются DEM модели (растровый набор данных, где каждый пиксель содержит высоту земной поверхности над уровнем моря).
Бесплатные источники таких данных доступны из нескольких глобальных проектов, причем большинство из них имеют открытые API для работы.

Получение информации о рельефе на суше доступно из проектов:

\begin{itemize}
\item SRTM (Shuttle Radar Topography Mission);
\item ASTER GDEM (Advanced Spaceborne Thermal Emission and Reflection Radi\-ometer);
\item ALOS World 3D (AW3D30);
\item FABDEM — Forest And Buildings DEM.
\end{itemize}

Во всех случаях пространственное разрешение моделей высот составляет порядка 30 м (1" по широте и долготе), что вполне достаточно для вычисления искомых характеристик.

Доступные глобальные батиметрические модели обладают более скромными параметрами и доступны из проекта GEBCO (General Bathymetric Chart of the Oceans) с пространственным разрешением 15".

\subsubsection{Параметры волн}

В большинстве работ по глобальной оценке волн на прибрежные территории используются глобальные источники данных:

\begin{itemize}
\item NOAA Wave Watch III (WW3) с разрешением 0.5° (примерно 55 км на экваторе);
\item ECMWF ERA5 Reanalysis (Copernicus) с разрешением 0.25° (примерно 28 км).
\end{itemize}

Однако опыт рассмотренных работ показал, что использовать данные с такой низкой плотностью ограниченно полезно.
К счастью методики расчета параметров волнового наката являются хорошо изученными и методически описанными в гидротехнике.
Пример такой методики утвержденной в РФ описан в СП 38.13330.2018 \enquote{Нагрузки и воздействия на гидротехнические сооружения (волновые, ледовые и от судов)} \cite{SP38_13330_2018}.

Исходными данными к выполняемым расчетам по данной методике являются батиметрия, положение береговой линии и усредненные диаграммы по розе ветров.
Если первые два параметра уже упоминались ранее как необходимые источники информации параметры ветров и погоды следует рассмотреть отдельно.

\vspace{1em}
\textbf{Получение метеорологических данных}
\vspace{1em}

В Российской Федерации основным ответственным органом за хранение метеорологических данных является Федеральная служба по гидрометеорологии и мониторингу окружающей среды (Росгидромет), которая формирует и ведет Единый государственный фонд данных о состоянии окружающей среды (ЕГФД).
Формально информация ЕГФД является открытой и общедоступной, за исключением сведений ограниченного доступа, но свободный прямой онлайн-доступ к полным архивным метеоданным отсутствует.

Наравне с этим в мире существует набор агрегаторов метео данных, предоставляющих доступ к своей информации через удобные API.
В рамках НИР были использованы 4 следующих открытых источника метеоданных:

\begin{itemize}
\item Open-Meteo -- данные включают исторические архивы (с 1940 г.) по следующим параметрам: температура, осадки, ветер, давление, облачность, UV-индекс и др. для 10 км пространственного разрешения;
\item Meteostat;
\item Visual Crossing Weather;
\item OpenWe\-atherMap.
\end{itemize}

Во всех сервисах присутствует возможность получения исторических данных о погоде, однако они ограниченны в бесплатном доступе (как в OpenWea\-ther\-Map) или фрагментарны (как в Meteostat).
Однако Meteostat прекрасно справляется с выдачей исторической статистики о работе метеостанциий по территориям во времени, историческое положение метеостанций может быть использовано при запросах в Visual Crossing Weather, позволяющему получить наиболее полную информацию, но ограниченное количество раз в сутки.

Наиболее перспективным сервисом же представляется Open-meteo, допускающий большее число дневных запросов, но выдающий данные только внутри растровой сетки с пространственным разрешением в 9 км.

Для автоматизации получения и сохранения данных был создан проект "Be\-ach\-Wea\-ther", перейти к которому можно по ссылке (рисунок \ref{pic:QR_code_BW}):

\begin{figure}
    \begin{center}
	\includegraphics[width=40mm]{fig/QR_code_BW.png}
	\caption{-- QR code на проект "BeachWeather"}
	\label{pic:QR_code_BW}
    \end{center}
\end{figure}

На текущий момент в нем реализованы базовые скрипты получения данных с упомянутых выше сервисов в формате  \verb|JSON| и их парсинга для сохранения в базе данных. 
Также была сформирована база данных (рисунок \ref{pic:БД_погода}) позволяющая накапливать в себе информацию для аккумуляции информации и демпфирования ограничений в доступных запросах.

\begin{figure}
    \begin{center}
	\includegraphics[width=\textwidth]{fig/БД_погода.png}
	\caption{-- Схема базы данных для агрегации метеорологических данных}
	\label{pic:БД_погода}
    \end{center}
\end{figure}

Структура разработанной базы данных для накопления метеорологических данных

\subsubsection{Скорость подъема уровня моря}

Скорость подъема уровня моря для расчета CVI можно получить из нескольких открытых источников:

\begin{itemize}
\item Permanent Service for Mean Sea Level — PSMSL;
\item Copernicus Marine Service (CMS);
\item ESA Climate Change Initiative (CCI).
\end{itemize}

PSMSL, предоставляющий данные об измерении уровня моря (мареографические данные) -- это крупнейшая глобальная база данных уровня моря с береговых мареографов.
Этот источник данных позволят проанализировать их на большем временном промежутке, однако сильно привязан к конкретным точкам положения наблюдательных станций.

Данные о глобальной альтиметрии уровня вод с большим пространственным разрешением доступны через сервисы CMS и CCI, но содержат данные с 1993 года, так как используют данные космического зондирования.
Пространственное разрешение этих данных 0.25°, что, с учетом малой скорости измерений, вполне достаточно для анализа. 

\subsubsection{Cкорость изменения береговой линии}

Параметры описывающие скорость изменения береговой линии являются наиболее труднодоступными и наименее надежными для анализа.
Наилучшие результаты достигаются при периодическом мониторинге геодезическими и фотограмметрическими методами, но они применимы только на локальных участках.

Альтернативой, при глобальном анализе, могут являться данные полученные при обработке радарных космических снимков.
Примеры выполненных на их основе расчетов сформированы в базах USGS National Shoreline Change (для побережий США) и European Marine Observation and Data Network (для территорий Европы).
Прибрежные территории РФ требуют отдельного расчета по методикам DSAS (Система цифрового анализа береговой линии), реализованной в пакете coastsat для python, однако необходимость обработки большого массива последовательных космических снимков систем Landsat 5/7/8/9 и Sentinel-2 скачивание которых затруднено в связи с санкционным режимом наложенным на РФ делает эту задачу труднореализуемой.

\subsection{Выводы по разделу}

Анализ доступных источников открытых данных показал, что есть все основания полагать, что поставленная в работе цель может быть достигнута на их основе.
Приведенный в работе обзор потенциальных источников требует дальнейшей проработки и подготовки специальных интерфейсов для своего получения, чему будет уделено основное внимание во время подготовки второго этапа НИР.
